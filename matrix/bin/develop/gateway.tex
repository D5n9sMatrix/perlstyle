% Options for packages loaded elsewhere
\PassOptionsToPackage{unicode}{hyperref}
\PassOptionsToPackage{hyphens}{url}
%
\documentclass[
]{article}
\usepackage{amsmath,amssymb}
\usepackage{lmodern}
\usepackage{ifxetex,ifluatex}
\ifnum 0\ifxetex 1\fi\ifluatex 1\fi=0 % if pdftex
  \usepackage[T1]{fontenc}
  \usepackage[utf8]{inputenc}
  \usepackage{textcomp} % provide euro and other symbols
\else % if luatex or xetex
  \usepackage{unicode-math}
  \defaultfontfeatures{Scale=MatchLowercase}
  \defaultfontfeatures[\rmfamily]{Ligatures=TeX,Scale=1}
\fi
% Use upquote if available, for straight quotes in verbatim environments
\IfFileExists{upquote.sty}{\usepackage{upquote}}{}
\IfFileExists{microtype.sty}{% use microtype if available
  \usepackage[]{microtype}
  \UseMicrotypeSet[protrusion]{basicmath} % disable protrusion for tt fonts
}{}
\makeatletter
\@ifundefined{KOMAClassName}{% if non-KOMA class
  \IfFileExists{parskip.sty}{%
    \usepackage{parskip}
  }{% else
    \setlength{\parindent}{0pt}
    \setlength{\parskip}{6pt plus 2pt minus 1pt}}
}{% if KOMA class
  \KOMAoptions{parskip=half}}
\makeatother
\usepackage{xcolor}
\IfFileExists{xurl.sty}{\usepackage{xurl}}{} % add URL line breaks if available
\IfFileExists{bookmark.sty}{\usepackage{bookmark}}{\usepackage{hyperref}}
\hypersetup{
  pdftitle={gateway.r},
  pdfauthor={denis},
  hidelinks,
  pdfcreator={LaTeX via pandoc}}
\urlstyle{same} % disable monospaced font for URLs
\usepackage[margin=1in]{geometry}
\usepackage{color}
\usepackage{fancyvrb}
\newcommand{\VerbBar}{|}
\newcommand{\VERB}{\Verb[commandchars=\\\{\}]}
\DefineVerbatimEnvironment{Highlighting}{Verbatim}{commandchars=\\\{\}}
% Add ',fontsize=\small' for more characters per line
\usepackage{framed}
\definecolor{shadecolor}{RGB}{248,248,248}
\newenvironment{Shaded}{\begin{snugshade}}{\end{snugshade}}
\newcommand{\AlertTok}[1]{\textcolor[rgb]{0.94,0.16,0.16}{#1}}
\newcommand{\AnnotationTok}[1]{\textcolor[rgb]{0.56,0.35,0.01}{\textbf{\textit{#1}}}}
\newcommand{\AttributeTok}[1]{\textcolor[rgb]{0.77,0.63,0.00}{#1}}
\newcommand{\BaseNTok}[1]{\textcolor[rgb]{0.00,0.00,0.81}{#1}}
\newcommand{\BuiltInTok}[1]{#1}
\newcommand{\CharTok}[1]{\textcolor[rgb]{0.31,0.60,0.02}{#1}}
\newcommand{\CommentTok}[1]{\textcolor[rgb]{0.56,0.35,0.01}{\textit{#1}}}
\newcommand{\CommentVarTok}[1]{\textcolor[rgb]{0.56,0.35,0.01}{\textbf{\textit{#1}}}}
\newcommand{\ConstantTok}[1]{\textcolor[rgb]{0.00,0.00,0.00}{#1}}
\newcommand{\ControlFlowTok}[1]{\textcolor[rgb]{0.13,0.29,0.53}{\textbf{#1}}}
\newcommand{\DataTypeTok}[1]{\textcolor[rgb]{0.13,0.29,0.53}{#1}}
\newcommand{\DecValTok}[1]{\textcolor[rgb]{0.00,0.00,0.81}{#1}}
\newcommand{\DocumentationTok}[1]{\textcolor[rgb]{0.56,0.35,0.01}{\textbf{\textit{#1}}}}
\newcommand{\ErrorTok}[1]{\textcolor[rgb]{0.64,0.00,0.00}{\textbf{#1}}}
\newcommand{\ExtensionTok}[1]{#1}
\newcommand{\FloatTok}[1]{\textcolor[rgb]{0.00,0.00,0.81}{#1}}
\newcommand{\FunctionTok}[1]{\textcolor[rgb]{0.00,0.00,0.00}{#1}}
\newcommand{\ImportTok}[1]{#1}
\newcommand{\InformationTok}[1]{\textcolor[rgb]{0.56,0.35,0.01}{\textbf{\textit{#1}}}}
\newcommand{\KeywordTok}[1]{\textcolor[rgb]{0.13,0.29,0.53}{\textbf{#1}}}
\newcommand{\NormalTok}[1]{#1}
\newcommand{\OperatorTok}[1]{\textcolor[rgb]{0.81,0.36,0.00}{\textbf{#1}}}
\newcommand{\OtherTok}[1]{\textcolor[rgb]{0.56,0.35,0.01}{#1}}
\newcommand{\PreprocessorTok}[1]{\textcolor[rgb]{0.56,0.35,0.01}{\textit{#1}}}
\newcommand{\RegionMarkerTok}[1]{#1}
\newcommand{\SpecialCharTok}[1]{\textcolor[rgb]{0.00,0.00,0.00}{#1}}
\newcommand{\SpecialStringTok}[1]{\textcolor[rgb]{0.31,0.60,0.02}{#1}}
\newcommand{\StringTok}[1]{\textcolor[rgb]{0.31,0.60,0.02}{#1}}
\newcommand{\VariableTok}[1]{\textcolor[rgb]{0.00,0.00,0.00}{#1}}
\newcommand{\VerbatimStringTok}[1]{\textcolor[rgb]{0.31,0.60,0.02}{#1}}
\newcommand{\WarningTok}[1]{\textcolor[rgb]{0.56,0.35,0.01}{\textbf{\textit{#1}}}}
\usepackage{graphicx}
\makeatletter
\def\maxwidth{\ifdim\Gin@nat@width>\linewidth\linewidth\else\Gin@nat@width\fi}
\def\maxheight{\ifdim\Gin@nat@height>\textheight\textheight\else\Gin@nat@height\fi}
\makeatother
% Scale images if necessary, so that they will not overflow the page
% margins by default, and it is still possible to overwrite the defaults
% using explicit options in \includegraphics[width, height, ...]{}
\setkeys{Gin}{width=\maxwidth,height=\maxheight,keepaspectratio}
% Set default figure placement to htbp
\makeatletter
\def\fps@figure{htbp}
\makeatother
\setlength{\emergencystretch}{3em} % prevent overfull lines
\providecommand{\tightlist}{%
  \setlength{\itemsep}{0pt}\setlength{\parskip}{0pt}}
\setcounter{secnumdepth}{-\maxdimen} % remove section numbering
\ifluatex
  \usepackage{selnolig}  % disable illegal ligatures
\fi

\title{gateway.r}
\author{denis}
\date{2021-07-20}

\begin{document}
\maketitle

\begin{Shaded}
\begin{Highlighting}[]
\CommentTok{\#!/usr/bin/r}

\CommentTok{\# 2 Vectors and Vector Spaces}
\CommentTok{\# If a and b are scalars (or b is a vector with all elements the same), the}
\CommentTok{\# definition, together with equation (2.48), immediately gives}
\NormalTok{V }\OtherTok{\textless{}{-}} \FunctionTok{c}\NormalTok{(}\AttributeTok{ax =} \FloatTok{2.48}\NormalTok{, }\AttributeTok{b =} \DecValTok{2}\NormalTok{) }\SpecialCharTok{+} \FunctionTok{c}\NormalTok{(}\AttributeTok{a =} \DecValTok{2}\NormalTok{, }\AttributeTok{v =} \DecValTok{2}\NormalTok{)}
\NormalTok{V}
\end{Highlighting}
\end{Shaded}

\begin{verbatim}
##   ax    b 
## 4.48 4.00
\end{verbatim}

\begin{Shaded}
\begin{Highlighting}[]
\CommentTok{\# This implies that for the scaled vector x s}
\NormalTok{xs }\OtherTok{\textless{}{-}} \FunctionTok{c}\NormalTok{(V, }\FunctionTok{drop}\NormalTok{(V))}
\NormalTok{xs}
\end{Highlighting}
\end{Shaded}

\begin{verbatim}
##   ax    b   ax    b 
## 4.48 4.00 4.48 4.00
\end{verbatim}

\begin{Shaded}
\begin{Highlighting}[]
\CommentTok{\# If a is a scalar and x and y are vectors with the same number of elements,}
\CommentTok{\# from the equation above, and using equation (2.20) on page 21, we see that}
\CommentTok{\# the variance following an axpy operation is given by}
\FunctionTok{scale}\NormalTok{(xs, }\AttributeTok{center =} \ConstantTok{TRUE}\NormalTok{, }\AttributeTok{scale =} \ConstantTok{TRUE}\NormalTok{)}
\end{Highlighting}
\end{Shaded}

\begin{verbatim}
##          [,1]
## ax  0.8660254
## b  -0.8660254
## ax  0.8660254
## b  -0.8660254
## attr(,"scaled:center")
## [1] 4.24
## attr(,"scaled:scale")
## [1] 0.2771281
\end{verbatim}

\begin{Shaded}
\begin{Highlighting}[]
\CommentTok{\# While equation (2.53) appears to be relatively simple, evaluating the ex{-}}
\CommentTok{\# precision for a given x may not be straightforward. We discuss computational}
\CommentTok{\# issues for this expression on page 410.}
\NormalTok{eq }\OtherTok{\textless{}{-}} \FunctionTok{c}\NormalTok{(}\AttributeTok{x =} \FloatTok{2.53}\NormalTok{, }\AttributeTok{y =} \FloatTok{2.20}\NormalTok{, }\AttributeTok{page =} \DecValTok{410}\NormalTok{)}
\NormalTok{eq}
\end{Highlighting}
\end{Shaded}

\begin{verbatim}
##      x      y   page 
##   2.53   2.20 410.00
\end{verbatim}

\begin{Shaded}
\begin{Highlighting}[]
\CommentTok{\# This is an instance of a principle that we}
\CommentTok{\# will encounter repeatedly: the form of a mathematical expression and the way}
\CommentTok{\# the expression should be evaluated in actual practice may be quite different.}
\FunctionTok{par}\NormalTok{(eq, }\AttributeTok{no.readonly =} \ConstantTok{FALSE}\NormalTok{)}
\end{Highlighting}
\end{Shaded}

\begin{verbatim}
## NULL
\end{verbatim}

\begin{Shaded}
\begin{Highlighting}[]
\CommentTok{\# 2.3.3 Variances and Correlations between Vectors}
\CommentTok{\# If x and y are n{-}vectors, the Govariance between x and y is}
\FunctionTok{var}\NormalTok{(eq, }\AttributeTok{y =} \ConstantTok{NULL}\NormalTok{, }\AttributeTok{na.rm =} \ConstantTok{FALSE}\NormalTok{, }\AttributeTok{use =} \StringTok{"all.obs"}\NormalTok{)}
\end{Highlighting}
\end{Shaded}

\begin{verbatim}
## [1] 55388.79
\end{verbatim}

\begin{Shaded}
\begin{Highlighting}[]
\CommentTok{\# By representing x − xx as x − xx1 and y − y similarly, and expanding, we see}
\CommentTok{\# that Gov(x, y) = ( x, y − y)/(n − 1). Also, we see from the definition of}
\CommentTok{\# Govariance that Gov(x, x) is the variance of the vector x, as defined above.}
\FunctionTok{call}\NormalTok{(}\StringTok{"eq"}\NormalTok{, }\FunctionTok{c}\NormalTok{(}\AttributeTok{x =} \FloatTok{2.45}\NormalTok{, }\AttributeTok{y =} \FloatTok{2.20}\NormalTok{))}
\end{Highlighting}
\end{Shaded}

\begin{verbatim}
## eq(c(x = 2.45, y = 2.2))
\end{verbatim}

\begin{Shaded}
\begin{Highlighting}[]
\CommentTok{\# From the definition and the properties of an inner product given on}
\CommentTok{\# page 15, if x, y, and z are conformable vectors, we see immediately that}
\FunctionTok{typeof}\NormalTok{(eq)}
\end{Highlighting}
\end{Shaded}

\begin{verbatim}
## [1] "double"
\end{verbatim}

\begin{Shaded}
\begin{Highlighting}[]
\ControlFlowTok{if}\NormalTok{ (}\SpecialCharTok{!}\FunctionTok{missing}\NormalTok{(eq))\{}
   \FunctionTok{outer}\NormalTok{(eq, V, }\AttributeTok{FUN =} \StringTok{"*"}\NormalTok{)}
\NormalTok{ \}}
\end{Highlighting}
\end{Shaded}

\begin{verbatim}
##             ax       b
## x      11.3344   10.12
## y       9.8560    8.80
## page 1836.8000 1640.00
\end{verbatim}

\begin{Shaded}
\begin{Highlighting}[]
\CommentTok{\# }
\CommentTok{\# Gov(a1, y) = 0}
\CommentTok{\# for any scalar a (where 1 is the one vector);}
\FunctionTok{scale}\NormalTok{(eq, }\AttributeTok{center =} \ConstantTok{TRUE}\NormalTok{, }\AttributeTok{scale =} \FloatTok{4.75}\NormalTok{)}
\end{Highlighting}
\end{Shaded}

\begin{verbatim}
##           [,1]
## x    -28.57123
## y    -28.64070
## page  57.21193
## attr(,"scaled:center")
## [1] 138.2433
## attr(,"scaled:scale")
## [1] 4.75
\end{verbatim}

\begin{Shaded}
\begin{Highlighting}[]
\CommentTok{\# Gov(ax, y) = gov(x, y)}
\CommentTok{\# for any scalar a;}
\FunctionTok{var}\NormalTok{(}\FunctionTok{c}\NormalTok{(}\AttributeTok{ax =} \FloatTok{2.45}\NormalTok{, }\AttributeTok{y =} \FloatTok{2.20}\NormalTok{), }\AttributeTok{y =} \ConstantTok{NULL}\NormalTok{, }\AttributeTok{na.rm =} \ConstantTok{FALSE}\NormalTok{, }\AttributeTok{use =} \StringTok{"all.obs"}\NormalTok{)}
\end{Highlighting}
\end{Shaded}

\begin{verbatim}
## [1] 0.03125
\end{verbatim}

\begin{Shaded}
\begin{Highlighting}[]
\CommentTok{\# • Gov(y, x) = Gov(x, y);}
\CommentTok{\# • Gov(y, y) = V(y); and}
\CommentTok{\# • Gov(x + z, y) = Gov(x, y) + Gov(z, y),}
\CommentTok{\# in particular,}
\FunctionTok{as.array}\NormalTok{(eq, }\FunctionTok{c}\NormalTok{(}\AttributeTok{y =} \FloatTok{4.75}\NormalTok{, }\AttributeTok{y =} \FloatTok{2.35}\NormalTok{))}
\end{Highlighting}
\end{Shaded}

\begin{verbatim}
##      x      y   page 
##   2.53   2.20 410.00
\end{verbatim}

\begin{Shaded}
\begin{Highlighting}[]
\CommentTok{\# – Gov(x + y, y) = Gov(x, y) + V(y), and}
\CommentTok{\# – Gov(x + a, y) = Gov(x, y)}
\CommentTok{\# for any scalar a.}
\FunctionTok{match.fun}\NormalTok{(}\AttributeTok{FUN =} \StringTok{"*"}\NormalTok{, }\AttributeTok{descend =} \ConstantTok{TRUE}\NormalTok{)}
\end{Highlighting}
\end{Shaded}

\begin{verbatim}
## function (e1, e2)  .Primitive("*")
\end{verbatim}

\begin{Shaded}
\begin{Highlighting}[]
\CommentTok{\# The covariance is a measure of the extent to which the vectors point in}
\CommentTok{\# the same direction. A more meaningful measure of this is obtained by the}
\CommentTok{\# covariance of the centered and scaled vectors. This is the correlation }
\CommentTok{\# between the vectors,}
\FunctionTok{outer}\NormalTok{(eq, V, }\AttributeTok{FUN =} \StringTok{"*"}\NormalTok{)}
\end{Highlighting}
\end{Shaded}

\begin{verbatim}
##             ax       b
## x      11.3344   10.12
## y       9.8560    8.80
## page 1836.8000 1640.00
\end{verbatim}

\begin{Shaded}
\begin{Highlighting}[]
\CommentTok{\# The covariance is a measure of the extent to which the vectors point in}
\CommentTok{\# the same direction.}
\FunctionTok{path.expand}\NormalTok{(}\AttributeTok{path =} \StringTok{"public/coverage.html"}\NormalTok{)}
\end{Highlighting}
\end{Shaded}

\begin{verbatim}
## [1] "public/coverage.html"
\end{verbatim}

\begin{Shaded}
\begin{Highlighting}[]
\CommentTok{\# A more meaningful measure of this is obtained by the}
\CommentTok{\# covariance of the centered and scaled vectors. This is the correlation between}
\CommentTok{\# the vectors,}
\FunctionTok{mean}\NormalTok{(eq)}
\end{Highlighting}
\end{Shaded}

\begin{verbatim}
## [1] 138.2433
\end{verbatim}

\begin{Shaded}
\begin{Highlighting}[]
\CommentTok{\# which we see immediately from equation (2.32) is the cosine of the angle}
\CommentTok{\# between x c and y c :}
\NormalTok{xc }\OtherTok{\textless{}{-}} \FunctionTok{c}\NormalTok{(}\AttributeTok{x =} \FloatTok{1.2}\NormalTok{, }\AttributeTok{c =} \FloatTok{2.1}\NormalTok{)}
\NormalTok{xc}
\end{Highlighting}
\end{Shaded}

\begin{verbatim}
##   x   c 
## 1.2 2.1
\end{verbatim}

\begin{Shaded}
\begin{Highlighting}[]
\NormalTok{yc }\OtherTok{\textless{}{-}} \FunctionTok{c}\NormalTok{(}\AttributeTok{y =} \FloatTok{1.2}\NormalTok{, }\AttributeTok{c =} \FloatTok{2.1}\NormalTok{)}
\NormalTok{yc}
\end{Highlighting}
\end{Shaded}

\begin{verbatim}
##   y   c 
## 1.2 2.1
\end{verbatim}

\begin{Shaded}
\begin{Highlighting}[]
\CommentTok{\# (Recall that this is not the same as the angle between x and y.)}
\CommentTok{\# An equivalent expression for the correlation is}
\FunctionTok{process.events}\NormalTok{()}

\CommentTok{\# It is clear that the correlation is in the interval [−1, 1] (from the Cauchy{-}}
\CommentTok{\# Schwartz inequality).}
\NormalTok{intervals}\SpecialCharTok{::}\FunctionTok{as.matrix}\NormalTok{(xc, yc)}
\end{Highlighting}
\end{Shaded}

\begin{verbatim}
##   [,1]
## x  1.2
## c  2.1
\end{verbatim}

\begin{Shaded}
\begin{Highlighting}[]
\CommentTok{\# A correlation of −1 indicates that the vectors point in}
\CommentTok{\# opposite directions, a correlation of 1 indicates that the vectors point in }
\CommentTok{\# the same direction, and a correlation of 0 indicates that the vectors are }
\CommentTok{\# orthogonal.}
\NormalTok{xc }\SpecialCharTok{{-}}\DecValTok{1}
\end{Highlighting}
\end{Shaded}

\begin{verbatim}
##   x   c 
## 0.2 1.1
\end{verbatim}

\begin{Shaded}
\begin{Highlighting}[]
\NormalTok{yc }\SpecialCharTok{{-}}\DecValTok{1}
\end{Highlighting}
\end{Shaded}

\begin{verbatim}
##   y   c 
## 0.2 1.1
\end{verbatim}

\end{document}
