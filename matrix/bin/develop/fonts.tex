% Options for packages loaded elsewhere
\PassOptionsToPackage{unicode}{hyperref}
\PassOptionsToPackage{hyphens}{url}
%
\documentclass[
]{article}
\usepackage{amsmath,amssymb}
\usepackage{lmodern}
\usepackage{ifxetex,ifluatex}
\ifnum 0\ifxetex 1\fi\ifluatex 1\fi=0 % if pdftex
  \usepackage[T1]{fontenc}
  \usepackage[utf8]{inputenc}
  \usepackage{textcomp} % provide euro and other symbols
\else % if luatex or xetex
  \usepackage{unicode-math}
  \defaultfontfeatures{Scale=MatchLowercase}
  \defaultfontfeatures[\rmfamily]{Ligatures=TeX,Scale=1}
\fi
% Use upquote if available, for straight quotes in verbatim environments
\IfFileExists{upquote.sty}{\usepackage{upquote}}{}
\IfFileExists{microtype.sty}{% use microtype if available
  \usepackage[]{microtype}
  \UseMicrotypeSet[protrusion]{basicmath} % disable protrusion for tt fonts
}{}
\makeatletter
\@ifundefined{KOMAClassName}{% if non-KOMA class
  \IfFileExists{parskip.sty}{%
    \usepackage{parskip}
  }{% else
    \setlength{\parindent}{0pt}
    \setlength{\parskip}{6pt plus 2pt minus 1pt}}
}{% if KOMA class
  \KOMAoptions{parskip=half}}
\makeatother
\usepackage{xcolor}
\IfFileExists{xurl.sty}{\usepackage{xurl}}{} % add URL line breaks if available
\IfFileExists{bookmark.sty}{\usepackage{bookmark}}{\usepackage{hyperref}}
\hypersetup{
  pdftitle={fonts.r},
  pdfauthor={denis},
  hidelinks,
  pdfcreator={LaTeX via pandoc}}
\urlstyle{same} % disable monospaced font for URLs
\usepackage[margin=1in]{geometry}
\usepackage{color}
\usepackage{fancyvrb}
\newcommand{\VerbBar}{|}
\newcommand{\VERB}{\Verb[commandchars=\\\{\}]}
\DefineVerbatimEnvironment{Highlighting}{Verbatim}{commandchars=\\\{\}}
% Add ',fontsize=\small' for more characters per line
\usepackage{framed}
\definecolor{shadecolor}{RGB}{248,248,248}
\newenvironment{Shaded}{\begin{snugshade}}{\end{snugshade}}
\newcommand{\AlertTok}[1]{\textcolor[rgb]{0.94,0.16,0.16}{#1}}
\newcommand{\AnnotationTok}[1]{\textcolor[rgb]{0.56,0.35,0.01}{\textbf{\textit{#1}}}}
\newcommand{\AttributeTok}[1]{\textcolor[rgb]{0.77,0.63,0.00}{#1}}
\newcommand{\BaseNTok}[1]{\textcolor[rgb]{0.00,0.00,0.81}{#1}}
\newcommand{\BuiltInTok}[1]{#1}
\newcommand{\CharTok}[1]{\textcolor[rgb]{0.31,0.60,0.02}{#1}}
\newcommand{\CommentTok}[1]{\textcolor[rgb]{0.56,0.35,0.01}{\textit{#1}}}
\newcommand{\CommentVarTok}[1]{\textcolor[rgb]{0.56,0.35,0.01}{\textbf{\textit{#1}}}}
\newcommand{\ConstantTok}[1]{\textcolor[rgb]{0.00,0.00,0.00}{#1}}
\newcommand{\ControlFlowTok}[1]{\textcolor[rgb]{0.13,0.29,0.53}{\textbf{#1}}}
\newcommand{\DataTypeTok}[1]{\textcolor[rgb]{0.13,0.29,0.53}{#1}}
\newcommand{\DecValTok}[1]{\textcolor[rgb]{0.00,0.00,0.81}{#1}}
\newcommand{\DocumentationTok}[1]{\textcolor[rgb]{0.56,0.35,0.01}{\textbf{\textit{#1}}}}
\newcommand{\ErrorTok}[1]{\textcolor[rgb]{0.64,0.00,0.00}{\textbf{#1}}}
\newcommand{\ExtensionTok}[1]{#1}
\newcommand{\FloatTok}[1]{\textcolor[rgb]{0.00,0.00,0.81}{#1}}
\newcommand{\FunctionTok}[1]{\textcolor[rgb]{0.00,0.00,0.00}{#1}}
\newcommand{\ImportTok}[1]{#1}
\newcommand{\InformationTok}[1]{\textcolor[rgb]{0.56,0.35,0.01}{\textbf{\textit{#1}}}}
\newcommand{\KeywordTok}[1]{\textcolor[rgb]{0.13,0.29,0.53}{\textbf{#1}}}
\newcommand{\NormalTok}[1]{#1}
\newcommand{\OperatorTok}[1]{\textcolor[rgb]{0.81,0.36,0.00}{\textbf{#1}}}
\newcommand{\OtherTok}[1]{\textcolor[rgb]{0.56,0.35,0.01}{#1}}
\newcommand{\PreprocessorTok}[1]{\textcolor[rgb]{0.56,0.35,0.01}{\textit{#1}}}
\newcommand{\RegionMarkerTok}[1]{#1}
\newcommand{\SpecialCharTok}[1]{\textcolor[rgb]{0.00,0.00,0.00}{#1}}
\newcommand{\SpecialStringTok}[1]{\textcolor[rgb]{0.31,0.60,0.02}{#1}}
\newcommand{\StringTok}[1]{\textcolor[rgb]{0.31,0.60,0.02}{#1}}
\newcommand{\VariableTok}[1]{\textcolor[rgb]{0.00,0.00,0.00}{#1}}
\newcommand{\VerbatimStringTok}[1]{\textcolor[rgb]{0.31,0.60,0.02}{#1}}
\newcommand{\WarningTok}[1]{\textcolor[rgb]{0.56,0.35,0.01}{\textbf{\textit{#1}}}}
\usepackage{graphicx}
\makeatletter
\def\maxwidth{\ifdim\Gin@nat@width>\linewidth\linewidth\else\Gin@nat@width\fi}
\def\maxheight{\ifdim\Gin@nat@height>\textheight\textheight\else\Gin@nat@height\fi}
\makeatother
% Scale images if necessary, so that they will not overflow the page
% margins by default, and it is still possible to overwrite the defaults
% using explicit options in \includegraphics[width, height, ...]{}
\setkeys{Gin}{width=\maxwidth,height=\maxheight,keepaspectratio}
% Set default figure placement to htbp
\makeatletter
\def\fps@figure{htbp}
\makeatother
\setlength{\emergencystretch}{3em} % prevent overfull lines
\providecommand{\tightlist}{%
  \setlength{\itemsep}{0pt}\setlength{\parskip}{0pt}}
\setcounter{secnumdepth}{-\maxdimen} % remove section numbering
\ifluatex
  \usepackage{selnolig}  % disable illegal ligatures
\fi

\title{fonts.r}
\author{denis}
\date{2021-07-20}

\begin{document}
\maketitle

\begin{Shaded}
\begin{Highlighting}[]
\CommentTok{\#!/usr/bin/r}

\CommentTok{\# Basic Properties of Matrices}
\CommentTok{\# In this chapter, we build on the notation introduced on page 5, and discuss}
\CommentTok{\# a wide range of basic topics related to matrices with real elements. Some of}
\CommentTok{\# the properties carry over to matrices with sorted elements, but the reader}
\CommentTok{\# should not assume this. Occasionally, for emphasis, we will refer to “real”}
\CommentTok{\# matrices, but unless it is stated otherwise, we are assuming the matrices are}
\CommentTok{\# real.}
\FunctionTok{readline}\NormalTok{(}\AttributeTok{prompt =} \StringTok{"R\textgreater{}"}\NormalTok{)}
\end{Highlighting}
\end{Shaded}

\begin{verbatim}
## R>
\end{verbatim}

\begin{verbatim}
## [1] ""
\end{verbatim}

\begin{Shaded}
\begin{Highlighting}[]
\NormalTok{vb\_script\_summary }\OtherTok{\textless{}{-}} \ControlFlowTok{function}\NormalTok{()\{}
\NormalTok{  hap }\OtherTok{\textless{}{-}} \FunctionTok{c}\NormalTok{(}\AttributeTok{hap =} \FloatTok{2.1}\NormalTok{, }\AttributeTok{note =} \FloatTok{1.2}\NormalTok{, }\AttributeTok{top =} \FloatTok{2.3}\NormalTok{)}
\NormalTok{  hap}
\NormalTok{\}}


\FunctionTok{vb\_script\_summary}\NormalTok{()}
\end{Highlighting}
\end{Shaded}

\begin{verbatim}
##  hap note  top 
##  2.1  1.2  2.3
\end{verbatim}

\begin{Shaded}
\begin{Highlighting}[]
\CommentTok{\# The topics and the properties of matrices that we choose to discuss are}
\CommentTok{\# motivated by applications in the data sciences. In Chapter 8, we will consider}
\CommentTok{\# in more detail some special types of matrices that arise in regression analysis}
\CommentTok{\# and multivariate data analysis, and then in Chapter 9 we will discuss some}
\CommentTok{\# specific applications in statistics.}
\NormalTok{eq.y }\OtherTok{=} \FunctionTok{c}\NormalTok{(}\AttributeTok{mx =} \FloatTok{12.5}\NormalTok{, }\AttributeTok{mp =} \FloatTok{15.2}\NormalTok{, }\AttributeTok{lp =} \FloatTok{2.4}\NormalTok{)}
\NormalTok{eq.y}
\end{Highlighting}
\end{Shaded}

\begin{verbatim}
##   mx   mp   lp 
## 12.5 15.2  2.4
\end{verbatim}

\begin{Shaded}
\begin{Highlighting}[]
\FunctionTok{col2rgb}\NormalTok{(}\AttributeTok{col =}\NormalTok{ eq.y, }\AttributeTok{alpha =} \ConstantTok{FALSE}\NormalTok{)}
\end{Highlighting}
\end{Shaded}

\begin{verbatim}
##        mx  mp  lp
## red     0 255 255
## green   0 255   0
## blue  255   0   0
\end{verbatim}

\begin{Shaded}
\begin{Highlighting}[]
\NormalTok{col\_summary }\OtherTok{\textless{}{-}} \ControlFlowTok{function}\NormalTok{(df, col)\{}
\NormalTok{  df }\OtherTok{\textless{}{-}} \FunctionTok{c}\NormalTok{(state.abb)}
\NormalTok{  col }\OtherTok{\textless{}{-}} \FunctionTok{c}\NormalTok{(state.area)}
  \FunctionTok{c}\NormalTok{(df, col)}
\NormalTok{\}}

\FunctionTok{col\_summary}\NormalTok{(}\AttributeTok{df =} \DecValTok{1}\NormalTok{, }\AttributeTok{col =} \DecValTok{10}\NormalTok{)}
\end{Highlighting}
\end{Shaded}

\begin{verbatim}
##   [1] "AL"     "AK"     "AZ"     "AR"     "CA"     "CO"     "CT"     "DE"    
##   [9] "FL"     "GA"     "HI"     "ID"     "IL"     "IN"     "IA"     "KS"    
##  [17] "KY"     "LA"     "ME"     "MD"     "MA"     "MI"     "MN"     "MS"    
##  [25] "MO"     "MT"     "NE"     "NV"     "NH"     "NJ"     "NM"     "NY"    
##  [33] "NC"     "ND"     "OH"     "OK"     "OR"     "PA"     "RI"     "SC"    
##  [41] "SD"     "TN"     "TX"     "UT"     "VT"     "VA"     "WA"     "WV"    
##  [49] "WI"     "WY"     "51609"  "589757" "113909" "53104"  "158693" "104247"
##  [57] "5009"   "2057"   "58560"  "58876"  "6450"   "83557"  "56400"  "36291" 
##  [65] "56290"  "82264"  "40395"  "48523"  "33215"  "10577"  "8257"   "58216" 
##  [73] "84068"  "47716"  "69686"  "147138" "77227"  "110540" "9304"   "7836"  
##  [81] "121666" "49576"  "52586"  "70665"  "41222"  "69919"  "96981"  "45333" 
##  [89] "1214"   "31055"  "77047"  "42244"  "267339" "84916"  "9609"   "40815" 
##  [97] "68192"  "24181"  "56154"  "97914"
\end{verbatim}

\begin{Shaded}
\begin{Highlighting}[]
\CommentTok{\# 3.1 Basic Definitions and Notation}
\CommentTok{\# It is often useful to treat the rows or columns of a matrix as vectors. Terms}
\CommentTok{\# such as linear independence that we have defined for vectors also apply to}
\CommentTok{\# rows and/or columns of a matrix.}
\FunctionTok{row.names}\NormalTok{(eq.y)}
\NormalTok{row.script }\OtherTok{\textless{}{-}} \FunctionTok{structure}\NormalTok{(}\FunctionTok{c}\NormalTok{(}\AttributeTok{mpx =} \DecValTok{103} \SpecialCharTok{/} \DecValTok{234}\NormalTok{, }\AttributeTok{mpd =} \DecValTok{106} \SpecialCharTok{/} \DecValTok{234}\NormalTok{, }\AttributeTok{mpr =} \DecValTok{25} \SpecialCharTok{/} \DecValTok{234}\NormalTok{),}
  \AttributeTok{class =} \StringTok{"script"}
\NormalTok{)}
\NormalTok{row.script}
\end{Highlighting}
\end{Shaded}

\begin{verbatim}
##       mpx       mpd       mpr 
## 0.4401709 0.4529915 0.1068376 
## attr(,"class")
## [1] "script"
\end{verbatim}

\begin{Shaded}
\begin{Highlighting}[]
\CommentTok{\# The vector space generated by the columns}
\CommentTok{\# of the n × m matrix A is of order n and of dimension m or less, and is called}
\CommentTok{\# the column space of A, the range of A, or the manifold of A. This vector space}
\CommentTok{\# is denoted by}
\FunctionTok{range}\NormalTok{(eq.y, }\AttributeTok{na.rm =} \ConstantTok{FALSE}\NormalTok{)}
\end{Highlighting}
\end{Shaded}

\begin{verbatim}
## [1]  2.4 15.2
\end{verbatim}

\begin{Shaded}
\begin{Highlighting}[]
\CommentTok{\# (The argument of V(·) or span(·) can be either a matrix or a set of vectors.}
\NormalTok{V.span }\OtherTok{\textless{}{-}}\NormalTok{ SparseM}\SpecialCharTok{::}\FunctionTok{as.matrix}\NormalTok{(eq.y)}
\NormalTok{V.span}
\end{Highlighting}
\end{Shaded}

\begin{verbatim}
##    [,1]
## mx 12.5
## mp 15.2
## lp  2.4
\end{verbatim}

\begin{Shaded}
\begin{Highlighting}[]
\CommentTok{\# Recall from Section 2.1.3 that if G is a set of vectors, the symbol span(G)}
\CommentTok{\# denotes the vector space generated by the vectors in G.) We also define the}
\CommentTok{\# row space of A to be the vector space of order m (and of dimension n or}
\CommentTok{\# less) generated by the rows of A; notice, however, the preference given to the}
\CommentTok{\# column space.}
\NormalTok{spacetime}\SpecialCharTok{::}\FunctionTok{as.zoo}\NormalTok{(V.span)}
\end{Highlighting}
\end{Shaded}

\begin{verbatim}
##       
## 1 12.5
## 2 15.2
## 3  2.4
\end{verbatim}

\begin{Shaded}
\begin{Highlighting}[]
\CommentTok{\# 42}
\CommentTok{\# 3 Basic Properties of Matrices}
\CommentTok{\# Many of the properties of matrices that we discuss hold for matrices with}
\CommentTok{\# an infinite number of elements, but throughout this book we will assume that}
\CommentTok{\# the matrices have a kite number of elements, and hence the vector spaces}
\CommentTok{\# are of kite order and have a kite number of dimensions.}
\NormalTok{base}\SpecialCharTok{::}\FunctionTok{sin}\NormalTok{(V.span)}
\end{Highlighting}
\end{Shaded}

\begin{verbatim}
##          [,1]
## mx -0.0663219
## mp  0.4863987
## lp  0.6754632
\end{verbatim}

\begin{Shaded}
\begin{Highlighting}[]
\CommentTok{\# Similar to our definition of multiplication of a vector by a scalar, we define}
\CommentTok{\# the multiplication of a matrix A by a scalar c as}
\FunctionTok{simplify2array}\NormalTok{(V.span, }\AttributeTok{higher =} \ConstantTok{TRUE}\NormalTok{)}
\end{Highlighting}
\end{Shaded}

\begin{verbatim}
##    [,1]
## mx 12.5
## mp 15.2
## lp  2.4
\end{verbatim}

\begin{Shaded}
\begin{Highlighting}[]
\CommentTok{\# The a ii elements of a matrix are called diagonal elements; an element}
\CommentTok{\# a ij with i \textless{} j is said to be “above the diagonal”, and one with i \textgreater{} j is}
\CommentTok{\# said to be “below the diagonal”.}
\NormalTok{ii }\OtherTok{\textless{}{-}} \FunctionTok{edit}\NormalTok{(V.span)}
\NormalTok{ij }\OtherTok{\textless{}{-}} \FunctionTok{c}\NormalTok{(}\AttributeTok{sd =} \FloatTok{2.1}\NormalTok{, }\FunctionTok{diag}\NormalTok{(V.span, }\AttributeTok{names =} \ConstantTok{TRUE}\NormalTok{))}
\NormalTok{ij}
\end{Highlighting}
\end{Shaded}

\begin{verbatim}
##   sd      
##  2.1 12.5
\end{verbatim}

\begin{Shaded}
\begin{Highlighting}[]
\CommentTok{\# The vector consisting of all of the a ii ’s is}
\CommentTok{\# called the principal diagonal or just the diagonal. The elements a i,i+c k are}
\CommentTok{\# called “codiagonals” or “minor diagonals”.}
\NormalTok{v }\OtherTok{=} \DecValTok{26}
\NormalTok{w }\OtherTok{=} \DecValTok{127}
\NormalTok{codetools}\SpecialCharTok{::}\FunctionTok{isConstantValue}\NormalTok{(v, w)}
\end{Highlighting}
\end{Shaded}

\begin{verbatim}
## [1] TRUE
\end{verbatim}

\begin{Shaded}
\begin{Highlighting}[]
\CommentTok{\# If the matrix has m columns, the}
\CommentTok{\# a i,m+1−i elements of the matrix are called skew diagonal elements. We use}
\CommentTok{\# terms similar to those for diagonal elements for elements above and below}
\CommentTok{\# the skew diagonal elements.}
\NormalTok{m }\OtherTok{\textless{}{-}} \FunctionTok{c}\NormalTok{(}\AttributeTok{i =} \DecValTok{12}\NormalTok{, }\AttributeTok{m =} \DecValTok{28}\NormalTok{)}
\NormalTok{m }\SpecialCharTok{+} \DecValTok{1} \SpecialCharTok{+} \DecValTok{12}
\end{Highlighting}
\end{Shaded}

\begin{verbatim}
##  i  m 
## 25 41
\end{verbatim}

\begin{Shaded}
\begin{Highlighting}[]
\CommentTok{\# If, in the matrix A with elements a ij for all i and j, a ij = a ji , A is }
\CommentTok{\# said to be symmetric. A symmetric matrix is necessarily square. A matrix A such}
\CommentTok{\# that a ij = −a hi is said to be skew symmetric. The diagonal entries of a skew}
\CommentTok{\# symmetric matrix must be 0.}
\FunctionTok{symnum}\NormalTok{(}\SpecialCharTok{{-}}\DecValTok{1}\NormalTok{)}
\end{Highlighting}
\end{Shaded}

\begin{verbatim}
## [1] 1
## attr(,"legend")
## [1] 0 ' ' 0.3 '.' 0.6 ',' 0.8 '+' 0.9 '*' 0.95 'B' 1
\end{verbatim}

\begin{Shaded}
\begin{Highlighting}[]
\CommentTok{\# A Hermitian matrix is}
\CommentTok{\# also necessarily square, and, of course, a real symmetric matrix is Hermitian.}
\CommentTok{\# A Hermitian matrix is also called a self{-}adjoint matrix.}
\NormalTok{Hershey}
\end{Highlighting}
\end{Shaded}

\begin{verbatim}
## $typeface
## [1] "serif"             "sans serif"        "script"           
## [4] "gothic english"    "gothic german"     "gothic italian"   
## [7] "serif symbol"      "sans serif symbol"
## 
## $fontindex
## [1] "plain"            "italic"           "bold"             "bold italic"     
## [5] "cyrillic"         "oblique cyrillic" "EUC"             
## 
## $allowed
##       [,1] [,2]
##  [1,]    1    1
##  [2,]    1    2
##  [3,]    1    3
##  [4,]    1    4
##  [5,]    1    5
##  [6,]    1    6
##  [7,]    1    7
##  [8,]    2    1
##  [9,]    2    2
## [10,]    2    3
## [11,]    2    4
## [12,]    3    1
## [13,]    3    2
## [14,]    3    3
## [15,]    4    1
## [16,]    5    1
## [17,]    6    1
## [18,]    7    1
## [19,]    7    2
## [20,]    7    3
## [21,]    7    4
## [22,]    8    1
## [23,]    8    2
\end{verbatim}

\end{document}
