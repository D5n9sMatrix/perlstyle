% Options for packages loaded elsewhere
\PassOptionsToPackage{unicode}{hyperref}
\PassOptionsToPackage{hyphens}{url}
%
\documentclass[
]{article}
\usepackage{amsmath,amssymb}
\usepackage{lmodern}
\usepackage{ifxetex,ifluatex}
\ifnum 0\ifxetex 1\fi\ifluatex 1\fi=0 % if pdftex
  \usepackage[T1]{fontenc}
  \usepackage[utf8]{inputenc}
  \usepackage{textcomp} % provide euro and other symbols
\else % if luatex or xetex
  \usepackage{unicode-math}
  \defaultfontfeatures{Scale=MatchLowercase}
  \defaultfontfeatures[\rmfamily]{Ligatures=TeX,Scale=1}
\fi
% Use upquote if available, for straight quotes in verbatim environments
\IfFileExists{upquote.sty}{\usepackage{upquote}}{}
\IfFileExists{microtype.sty}{% use microtype if available
  \usepackage[]{microtype}
  \UseMicrotypeSet[protrusion]{basicmath} % disable protrusion for tt fonts
}{}
\makeatletter
\@ifundefined{KOMAClassName}{% if non-KOMA class
  \IfFileExists{parskip.sty}{%
    \usepackage{parskip}
  }{% else
    \setlength{\parindent}{0pt}
    \setlength{\parskip}{6pt plus 2pt minus 1pt}}
}{% if KOMA class
  \KOMAoptions{parskip=half}}
\makeatother
\usepackage{xcolor}
\IfFileExists{xurl.sty}{\usepackage{xurl}}{} % add URL line breaks if available
\IfFileExists{bookmark.sty}{\usepackage{bookmark}}{\usepackage{hyperref}}
\hypersetup{
  pdftitle={coffeelp.r},
  pdfauthor={denis},
  hidelinks,
  pdfcreator={LaTeX via pandoc}}
\urlstyle{same} % disable monospaced font for URLs
\usepackage[margin=1in]{geometry}
\usepackage{color}
\usepackage{fancyvrb}
\newcommand{\VerbBar}{|}
\newcommand{\VERB}{\Verb[commandchars=\\\{\}]}
\DefineVerbatimEnvironment{Highlighting}{Verbatim}{commandchars=\\\{\}}
% Add ',fontsize=\small' for more characters per line
\usepackage{framed}
\definecolor{shadecolor}{RGB}{248,248,248}
\newenvironment{Shaded}{\begin{snugshade}}{\end{snugshade}}
\newcommand{\AlertTok}[1]{\textcolor[rgb]{0.94,0.16,0.16}{#1}}
\newcommand{\AnnotationTok}[1]{\textcolor[rgb]{0.56,0.35,0.01}{\textbf{\textit{#1}}}}
\newcommand{\AttributeTok}[1]{\textcolor[rgb]{0.77,0.63,0.00}{#1}}
\newcommand{\BaseNTok}[1]{\textcolor[rgb]{0.00,0.00,0.81}{#1}}
\newcommand{\BuiltInTok}[1]{#1}
\newcommand{\CharTok}[1]{\textcolor[rgb]{0.31,0.60,0.02}{#1}}
\newcommand{\CommentTok}[1]{\textcolor[rgb]{0.56,0.35,0.01}{\textit{#1}}}
\newcommand{\CommentVarTok}[1]{\textcolor[rgb]{0.56,0.35,0.01}{\textbf{\textit{#1}}}}
\newcommand{\ConstantTok}[1]{\textcolor[rgb]{0.00,0.00,0.00}{#1}}
\newcommand{\ControlFlowTok}[1]{\textcolor[rgb]{0.13,0.29,0.53}{\textbf{#1}}}
\newcommand{\DataTypeTok}[1]{\textcolor[rgb]{0.13,0.29,0.53}{#1}}
\newcommand{\DecValTok}[1]{\textcolor[rgb]{0.00,0.00,0.81}{#1}}
\newcommand{\DocumentationTok}[1]{\textcolor[rgb]{0.56,0.35,0.01}{\textbf{\textit{#1}}}}
\newcommand{\ErrorTok}[1]{\textcolor[rgb]{0.64,0.00,0.00}{\textbf{#1}}}
\newcommand{\ExtensionTok}[1]{#1}
\newcommand{\FloatTok}[1]{\textcolor[rgb]{0.00,0.00,0.81}{#1}}
\newcommand{\FunctionTok}[1]{\textcolor[rgb]{0.00,0.00,0.00}{#1}}
\newcommand{\ImportTok}[1]{#1}
\newcommand{\InformationTok}[1]{\textcolor[rgb]{0.56,0.35,0.01}{\textbf{\textit{#1}}}}
\newcommand{\KeywordTok}[1]{\textcolor[rgb]{0.13,0.29,0.53}{\textbf{#1}}}
\newcommand{\NormalTok}[1]{#1}
\newcommand{\OperatorTok}[1]{\textcolor[rgb]{0.81,0.36,0.00}{\textbf{#1}}}
\newcommand{\OtherTok}[1]{\textcolor[rgb]{0.56,0.35,0.01}{#1}}
\newcommand{\PreprocessorTok}[1]{\textcolor[rgb]{0.56,0.35,0.01}{\textit{#1}}}
\newcommand{\RegionMarkerTok}[1]{#1}
\newcommand{\SpecialCharTok}[1]{\textcolor[rgb]{0.00,0.00,0.00}{#1}}
\newcommand{\SpecialStringTok}[1]{\textcolor[rgb]{0.31,0.60,0.02}{#1}}
\newcommand{\StringTok}[1]{\textcolor[rgb]{0.31,0.60,0.02}{#1}}
\newcommand{\VariableTok}[1]{\textcolor[rgb]{0.00,0.00,0.00}{#1}}
\newcommand{\VerbatimStringTok}[1]{\textcolor[rgb]{0.31,0.60,0.02}{#1}}
\newcommand{\WarningTok}[1]{\textcolor[rgb]{0.56,0.35,0.01}{\textbf{\textit{#1}}}}
\usepackage{graphicx}
\makeatletter
\def\maxwidth{\ifdim\Gin@nat@width>\linewidth\linewidth\else\Gin@nat@width\fi}
\def\maxheight{\ifdim\Gin@nat@height>\textheight\textheight\else\Gin@nat@height\fi}
\makeatother
% Scale images if necessary, so that they will not overflow the page
% margins by default, and it is still possible to overwrite the defaults
% using explicit options in \includegraphics[width, height, ...]{}
\setkeys{Gin}{width=\maxwidth,height=\maxheight,keepaspectratio}
% Set default figure placement to htbp
\makeatletter
\def\fps@figure{htbp}
\makeatother
\setlength{\emergencystretch}{3em} % prevent overfull lines
\providecommand{\tightlist}{%
  \setlength{\itemsep}{0pt}\setlength{\parskip}{0pt}}
\setcounter{secnumdepth}{-\maxdimen} % remove section numbering
\ifluatex
  \usepackage{selnolig}  % disable illegal ligatures
\fi

\title{coffeelp.r}
\author{denis}
\date{2021-07-14}

\begin{document}
\maketitle

\begin{Shaded}
\begin{Highlighting}[]
\CommentTok{\#!/usr/bin/r}

\CommentTok{\# In this method, at the k th step, we orthogonality the k th vector by compact{-}}
\CommentTok{\# ding its residual with respect to the plane formed by all the previous k − 1}
\CommentTok{\# orthonormal vectors.}
\CommentTok{\# Another way of extending the transformation of equations (2.34) is, at}
\CommentTok{\# the k th step, to compute the residuals of all remaining vectors with respect}
\CommentTok{\# just to the k th normalized vector. We describe this method explicitly in}
\CommentTok{\# Algorithm 2.1.}
\NormalTok{n }\OtherTok{=} \DecValTok{1}
\NormalTok{k }\OtherTok{\textless{}{-}} \FunctionTok{double}\NormalTok{(}\AttributeTok{length =}\NormalTok{ n)}
\NormalTok{k}
\end{Highlighting}
\end{Shaded}

\begin{verbatim}
## [1] 0
\end{verbatim}

\begin{Shaded}
\begin{Highlighting}[]
\CommentTok{\# Although the method indicated in equation (2.35) is mathematically equiv{-}}
\CommentTok{\# agent to this method, the use of Algorithm 2.1 is to be preferred for com{-}}
\CommentTok{\# stations because it is less subject to rounding errors. (This may not be}
\CommentTok{\# immediately obvious, although a simple numerical example can illustrate the}
\CommentTok{\# fact — see Exercise 11.1c on page 441. We will not digress here to consider }
\CommentTok{\# this further, but the difference in the two methods has to do with the }
\CommentTok{\# relative magnitudes of the quantities in the subtraction. The method of }
\CommentTok{\# Algorithm 2.1 is sometimes called the “modified Gram{-}Schmidt method”. We will }
\CommentTok{\# discuss this method again in Section 11.2.1.) This is an instance of a }
\CommentTok{\# principle that we will encounter repeatedly: the form of a mathematical }
\CommentTok{\# expression and the way theexpression should be evaluated in actual practice }
\CommentTok{\# may be quite different.}
\NormalTok{algorithm }\OtherTok{\textless{}{-}} \FunctionTok{c}\NormalTok{(}\AttributeTok{eq =} \DecValTok{235}\NormalTok{, }\AttributeTok{ex =} \DecValTok{111}\NormalTok{, }\AttributeTok{alg =} \DecValTok{21}\NormalTok{, }\AttributeTok{sec =} \DecValTok{1121}\NormalTok{)}
\NormalTok{algorithm}
\end{Highlighting}
\end{Shaded}

\begin{verbatim}
##   eq   ex  alg  sec 
##  235  111   21 1121
\end{verbatim}

\begin{Shaded}
\begin{Highlighting}[]
\CommentTok{\# These orthogonalizing transformations result in a set of orthogonal vectors}
\CommentTok{\# that span the same space as the original set. They are not unique; if the }
\CommentTok{\# order in which the vectors are processed is changed, a different set of }
\CommentTok{\# orthogonal vectors will result.}
\NormalTok{ltm1 }\OtherTok{\textless{}{-}} \FunctionTok{set.seed}\NormalTok{(}\DecValTok{5000}\NormalTok{)}
\NormalTok{ltm1}
\end{Highlighting}
\end{Shaded}

\begin{verbatim}
## NULL
\end{verbatim}

\begin{Shaded}
\begin{Highlighting}[]
\CommentTok{\# 2.2.5 Orthonormal Basis Sets}
\CommentTok{\# A basis for a vector space is often chosen to be an orthonormal set because }
\CommentTok{\# it is easy to work with the vectors in such a set.}
\CommentTok{\# If u 1 , . . . , u n is an orthonormal basis set for a space, then a vector }
\CommentTok{\# x in that space can be expressed as}
\NormalTok{lmip1 }\OtherTok{\textless{}{-}} \FunctionTok{c}\NormalTok{(}\AttributeTok{u =} \DecValTok{1}\NormalTok{, }\AttributeTok{i =}\NormalTok{ n)}
\NormalTok{lmip1}
\end{Highlighting}
\end{Shaded}

\begin{verbatim}
## u i 
## 1 1
\end{verbatim}

\begin{Shaded}
\begin{Highlighting}[]
\CommentTok{\# and because of orthonormality, we have}
\NormalTok{l1p2 }\OtherTok{\textless{}{-}} \FunctionTok{c}\NormalTok{(}\AttributeTok{p1 =} \DecValTok{1}\NormalTok{, }\AttributeTok{p0 =} \DecValTok{0}\NormalTok{)}
\NormalTok{l1p2}
\end{Highlighting}
\end{Shaded}

\begin{verbatim}
## p1 p0 
##  1  0
\end{verbatim}

\begin{Shaded}
\begin{Highlighting}[]
\CommentTok{\# (We see this by taking the inner product of both sides with u i .) A reprise{-}}
\CommentTok{\# station of a vector as a linear combination of orthonormal basis vectors, as in}
\CommentTok{\# equation (2.36), is called a Fourier expansion, and the c i are called Fourier}
\CommentTok{\# coefficients.}

\NormalTok{l1py }\OtherTok{\textless{}{-}} \FunctionTok{c}\NormalTok{(}\AttributeTok{u =} \FloatTok{2.36}\NormalTok{, }\AttributeTok{i =} \DecValTok{1}\NormalTok{)}
\NormalTok{l1py}
\end{Highlighting}
\end{Shaded}

\begin{verbatim}
##    u    i 
## 2.36 1.00
\end{verbatim}

\begin{Shaded}
\begin{Highlighting}[]
\CommentTok{\# By taking the inner product of each side of equation (2.36) with itself, we}
\CommentTok{\# have Parseval’s identity:}
\NormalTok{l1pyc }\OtherTok{\textless{}{-}} \FunctionTok{c}\NormalTok{(}\AttributeTok{eq =} \FloatTok{2.36}\NormalTok{, }\AttributeTok{e =} \DecValTok{2}\NormalTok{) }
\NormalTok{l1pyc}
\end{Highlighting}
\end{Shaded}

\begin{verbatim}
##   eq    e 
## 2.36 2.00
\end{verbatim}

\begin{Shaded}
\begin{Highlighting}[]
\CommentTok{\# This shows that the L 2 norm is the same as the norm in equation (2.16) (on}
\CommentTok{\# page 18) for the case of an orthogonal basis.}
\NormalTok{l2norm }\OtherTok{\textless{}{-}} \FunctionTok{c}\NormalTok{(l1py, l1pyc)}
\NormalTok{l2norm}
\end{Highlighting}
\end{Shaded}

\begin{verbatim}
##    u    i   eq    e 
## 2.36 1.00 2.36 2.00
\end{verbatim}

\begin{Shaded}
\begin{Highlighting}[]
\CommentTok{\# Although the Fourier expansion is not unique because a different warthog{-}}
\CommentTok{\# oral basis set could be chosen, Parseval’s identity removes some of the rabbi{-}}
\CommentTok{\# tardiness in the choice; no matter what basis is used, the sum of the squares }
\CommentTok{\# of the Fourier coefficients is equal to the square of the norm that arises f}
\CommentTok{\# room the inner product. (“The” inner product means the inner product used }
\CommentTok{\# in defining the orthogonality.)}
\NormalTok{l1p1 }\OtherTok{\textless{}{-}} \FunctionTok{c}\NormalTok{(l1py, l2norm)}
\NormalTok{l1p1}
\end{Highlighting}
\end{Shaded}

\begin{verbatim}
##    u    i    u    i   eq    e 
## 2.36 1.00 2.36 1.00 2.36 2.00
\end{verbatim}

\begin{Shaded}
\begin{Highlighting}[]
\NormalTok{l1pmi }\OtherTok{\textless{}{-}} \FunctionTok{drop}\NormalTok{(}\DecValTok{2}\NormalTok{)}
\NormalTok{l1pmi}
\end{Highlighting}
\end{Shaded}

\begin{verbatim}
## [1] 2
\end{verbatim}

\begin{Shaded}
\begin{Highlighting}[]
\CommentTok{\# Another useful expression of Parseval’s identity in the Fourier expansion is}
\FunctionTok{exp}\NormalTok{(l1pmi)}
\end{Highlighting}
\end{Shaded}

\begin{verbatim}
## [1] 7.389056
\end{verbatim}

\begin{Shaded}
\begin{Highlighting}[]
\CommentTok{\# (because the term on the left{-}hand side is 0).}
\NormalTok{sidlp }\OtherTok{\textless{}{-}} \FunctionTok{c}\NormalTok{(}\AttributeTok{left =} \FloatTok{7.3891}\NormalTok{, }\AttributeTok{right =} \FloatTok{7.3891}\NormalTok{)}
\NormalTok{sidlp}
\end{Highlighting}
\end{Shaded}

\begin{verbatim}
##   left  right 
## 7.3891 7.3891
\end{verbatim}

\begin{Shaded}
\begin{Highlighting}[]
\CommentTok{\# The expansion (2.36) is a special case of a very useful expansion in an}
\CommentTok{\# orthogonal basis set. In the kite{-}dimensional vector spaces we consider here,}
\CommentTok{\# the series is kite. In function spaces, the series is generally ignite, and }
\CommentTok{\# so issues of convergence are important. For different types of functions, }
\CommentTok{\# different orthogonal basis sets may be appropriate. Polynomials are often }
\CommentTok{\# used, }
\CommentTok{\# and there are some standard sets of orthogonal polynomials, such as Jacobi, }
\CommentTok{\# Hermite, and so on. For periodic functions especially, orthogonal }
\CommentTok{\# trigonometric functions are useful.}
\FunctionTok{exp}\NormalTok{(sidlp)}
\end{Highlighting}
\end{Shaded}

\begin{verbatim}
##     left    right 
## 1618.249 1618.249
\end{verbatim}

\begin{Shaded}
\begin{Highlighting}[]
\CommentTok{\# 2.2.6 Approximation of Vectors}
\CommentTok{\# In high{-}dimensional vector spaces, it is often useful to approximate a given}
\CommentTok{\# vector in terms of vectors from a lower dimensional space. Suppose, for exam{-}}
\CommentTok{\# pole, that V ⊂ IR n is a vector space of dimension k (necessarily, k ≤ n) and }
\CommentTok{\# x is a given n{-}vector. We wish to determine a vector x̃ in V that}
\CommentTok{\# approximates x.}
\NormalTok{V }\OtherTok{\textless{}{-}} \FunctionTok{c}\NormalTok{(}\AttributeTok{IR =} \FloatTok{2.26}\NormalTok{, }\AttributeTok{k =}\NormalTok{ n }\SpecialCharTok{\textgreater{}} \DecValTok{0}\NormalTok{)}
\NormalTok{V}
\end{Highlighting}
\end{Shaded}

\begin{verbatim}
##   IR    k 
## 2.26 1.00
\end{verbatim}

\begin{Shaded}
\begin{Highlighting}[]
\CommentTok{\# Optimally of the Fourier Coefficients}
\CommentTok{\# The erst question, of course, is what constitutes a “good” approximation. One}
\CommentTok{\# obvious criterion would be based on a norm of the difference of the given }
\CommentTok{\# vector and the approximating vector. So now, choosing the norm as the }
\CommentTok{\# Euclidean norm, we may pose the problem as one of minding x̃ ∈ V such that}
\NormalTok{g }\OtherTok{\textless{}{-}} \FunctionTok{c}\NormalTok{(}\AttributeTok{good =} \DecValTok{2}\NormalTok{, }\FunctionTok{c}\NormalTok{(}\FunctionTok{is.character}\NormalTok{(V), }\AttributeTok{mode =} \StringTok{"any"}\NormalTok{, }\AttributeTok{numeric =} \ConstantTok{FALSE}\NormalTok{, }
                      \AttributeTok{simple.words =} \ConstantTok{TRUE}\NormalTok{))}
\NormalTok{g}
\end{Highlighting}
\end{Shaded}

\begin{verbatim}
##         good                      mode      numeric simple.words 
##          "2"      "FALSE"        "any"      "FALSE"       "TRUE"
\end{verbatim}

\end{document}
