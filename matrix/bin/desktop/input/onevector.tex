% Options for packages loaded elsewhere
\PassOptionsToPackage{unicode}{hyperref}
\PassOptionsToPackage{hyphens}{url}
%
\documentclass[
]{article}
\usepackage{amsmath,amssymb}
\usepackage{lmodern}
\usepackage{ifxetex,ifluatex}
\ifnum 0\ifxetex 1\fi\ifluatex 1\fi=0 % if pdftex
  \usepackage[T1]{fontenc}
  \usepackage[utf8]{inputenc}
  \usepackage{textcomp} % provide euro and other symbols
\else % if luatex or xetex
  \usepackage{unicode-math}
  \defaultfontfeatures{Scale=MatchLowercase}
  \defaultfontfeatures[\rmfamily]{Ligatures=TeX,Scale=1}
\fi
% Use upquote if available, for straight quotes in verbatim environments
\IfFileExists{upquote.sty}{\usepackage{upquote}}{}
\IfFileExists{microtype.sty}{% use microtype if available
  \usepackage[]{microtype}
  \UseMicrotypeSet[protrusion]{basicmath} % disable protrusion for tt fonts
}{}
\makeatletter
\@ifundefined{KOMAClassName}{% if non-KOMA class
  \IfFileExists{parskip.sty}{%
    \usepackage{parskip}
  }{% else
    \setlength{\parindent}{0pt}
    \setlength{\parskip}{6pt plus 2pt minus 1pt}}
}{% if KOMA class
  \KOMAoptions{parskip=half}}
\makeatother
\usepackage{xcolor}
\IfFileExists{xurl.sty}{\usepackage{xurl}}{} % add URL line breaks if available
\IfFileExists{bookmark.sty}{\usepackage{bookmark}}{\usepackage{hyperref}}
\hypersetup{
  pdftitle={onevector.r},
  pdfauthor={denis},
  hidelinks,
  pdfcreator={LaTeX via pandoc}}
\urlstyle{same} % disable monospaced font for URLs
\usepackage[margin=1in]{geometry}
\usepackage{color}
\usepackage{fancyvrb}
\newcommand{\VerbBar}{|}
\newcommand{\VERB}{\Verb[commandchars=\\\{\}]}
\DefineVerbatimEnvironment{Highlighting}{Verbatim}{commandchars=\\\{\}}
% Add ',fontsize=\small' for more characters per line
\usepackage{framed}
\definecolor{shadecolor}{RGB}{248,248,248}
\newenvironment{Shaded}{\begin{snugshade}}{\end{snugshade}}
\newcommand{\AlertTok}[1]{\textcolor[rgb]{0.94,0.16,0.16}{#1}}
\newcommand{\AnnotationTok}[1]{\textcolor[rgb]{0.56,0.35,0.01}{\textbf{\textit{#1}}}}
\newcommand{\AttributeTok}[1]{\textcolor[rgb]{0.77,0.63,0.00}{#1}}
\newcommand{\BaseNTok}[1]{\textcolor[rgb]{0.00,0.00,0.81}{#1}}
\newcommand{\BuiltInTok}[1]{#1}
\newcommand{\CharTok}[1]{\textcolor[rgb]{0.31,0.60,0.02}{#1}}
\newcommand{\CommentTok}[1]{\textcolor[rgb]{0.56,0.35,0.01}{\textit{#1}}}
\newcommand{\CommentVarTok}[1]{\textcolor[rgb]{0.56,0.35,0.01}{\textbf{\textit{#1}}}}
\newcommand{\ConstantTok}[1]{\textcolor[rgb]{0.00,0.00,0.00}{#1}}
\newcommand{\ControlFlowTok}[1]{\textcolor[rgb]{0.13,0.29,0.53}{\textbf{#1}}}
\newcommand{\DataTypeTok}[1]{\textcolor[rgb]{0.13,0.29,0.53}{#1}}
\newcommand{\DecValTok}[1]{\textcolor[rgb]{0.00,0.00,0.81}{#1}}
\newcommand{\DocumentationTok}[1]{\textcolor[rgb]{0.56,0.35,0.01}{\textbf{\textit{#1}}}}
\newcommand{\ErrorTok}[1]{\textcolor[rgb]{0.64,0.00,0.00}{\textbf{#1}}}
\newcommand{\ExtensionTok}[1]{#1}
\newcommand{\FloatTok}[1]{\textcolor[rgb]{0.00,0.00,0.81}{#1}}
\newcommand{\FunctionTok}[1]{\textcolor[rgb]{0.00,0.00,0.00}{#1}}
\newcommand{\ImportTok}[1]{#1}
\newcommand{\InformationTok}[1]{\textcolor[rgb]{0.56,0.35,0.01}{\textbf{\textit{#1}}}}
\newcommand{\KeywordTok}[1]{\textcolor[rgb]{0.13,0.29,0.53}{\textbf{#1}}}
\newcommand{\NormalTok}[1]{#1}
\newcommand{\OperatorTok}[1]{\textcolor[rgb]{0.81,0.36,0.00}{\textbf{#1}}}
\newcommand{\OtherTok}[1]{\textcolor[rgb]{0.56,0.35,0.01}{#1}}
\newcommand{\PreprocessorTok}[1]{\textcolor[rgb]{0.56,0.35,0.01}{\textit{#1}}}
\newcommand{\RegionMarkerTok}[1]{#1}
\newcommand{\SpecialCharTok}[1]{\textcolor[rgb]{0.00,0.00,0.00}{#1}}
\newcommand{\SpecialStringTok}[1]{\textcolor[rgb]{0.31,0.60,0.02}{#1}}
\newcommand{\StringTok}[1]{\textcolor[rgb]{0.31,0.60,0.02}{#1}}
\newcommand{\VariableTok}[1]{\textcolor[rgb]{0.00,0.00,0.00}{#1}}
\newcommand{\VerbatimStringTok}[1]{\textcolor[rgb]{0.31,0.60,0.02}{#1}}
\newcommand{\WarningTok}[1]{\textcolor[rgb]{0.56,0.35,0.01}{\textbf{\textit{#1}}}}
\usepackage{graphicx}
\makeatletter
\def\maxwidth{\ifdim\Gin@nat@width>\linewidth\linewidth\else\Gin@nat@width\fi}
\def\maxheight{\ifdim\Gin@nat@height>\textheight\textheight\else\Gin@nat@height\fi}
\makeatother
% Scale images if necessary, so that they will not overflow the page
% margins by default, and it is still possible to overwrite the defaults
% using explicit options in \includegraphics[width, height, ...]{}
\setkeys{Gin}{width=\maxwidth,height=\maxheight,keepaspectratio}
% Set default figure placement to htbp
\makeatletter
\def\fps@figure{htbp}
\makeatother
\setlength{\emergencystretch}{3em} % prevent overfull lines
\providecommand{\tightlist}{%
  \setlength{\itemsep}{0pt}\setlength{\parskip}{0pt}}
\setcounter{secnumdepth}{-\maxdimen} % remove section numbering
\ifluatex
  \usepackage{selnolig}  % disable illegal ligatures
\fi

\title{onevector.r}
\author{denis}
\date{2021-07-13}

\begin{document}
\maketitle

\begin{Shaded}
\begin{Highlighting}[]
\CommentTok{\#!/usr/bin/r}

\CommentTok{\# 22}
\CommentTok{\# 2 Vectors and Vector Spaces}
\CommentTok{\# is a normalized vector. Normalized vectors are sometimes referred to as “unit}
\CommentTok{\# vectors”, although we will generally reserve this term for a special kind of }
\CommentTok{\# normalized vector (see page 12). A normalized vector is also sometimes }
\CommentTok{\# referred o as a “normal vector”. I use “normalized vector” for a vector such }
\CommentTok{\# as x̃ in equation (2.21) and use the latter phrase to denote a vector that is}
\CommentTok{\# orthogonal to a subspace.}
\NormalTok{norms }\OtherTok{\textless{}{-}} \FunctionTok{vector}\NormalTok{(}\AttributeTok{mode =} \StringTok{"logical"}\NormalTok{, }\AttributeTok{length =}\NormalTok{ 0L)}
\NormalTok{norms}
\end{Highlighting}
\end{Shaded}

\begin{verbatim}
## logical(0)
\end{verbatim}

\begin{Shaded}
\begin{Highlighting}[]
\CommentTok{\# 2.1.7 Metrics and Distances}
\CommentTok{\# It is often useful to consider how far apart two vectors are; that is, the }
\CommentTok{\# “distance” between them. A reasonable distance measure would have to satisfy}
\CommentTok{\# certain requirements, such as being a nonnegative real number. A function ∆}
\CommentTok{\# that maps any two objects in a set S to IR is called a metric on S if, for all}
\CommentTok{\# x, y, and z in S, it satisfies the following three conditions:}
\NormalTok{Matrix}\SpecialCharTok{::}\FunctionTok{as.matrix}\NormalTok{(norms)}
\end{Highlighting}
\end{Shaded}

\begin{verbatim}
##      [,1]
\end{verbatim}

\begin{Shaded}
\begin{Highlighting}[]
\CommentTok{\# These conditions correspond in an intuitive manner to the properties we ex{-}}
\CommentTok{\# pect of a distance between objects.}
\FunctionTok{object.size}\NormalTok{(norms)}
\end{Highlighting}
\end{Shaded}

\begin{verbatim}
## 48 bytes
\end{verbatim}

\begin{Shaded}
\begin{Highlighting}[]
\CommentTok{\# Metrics Induced by Norms}
\CommentTok{\# If subtraction and a norm are defined for the elements of S, the most common}
\CommentTok{\# way of forming a metric is by using the norm. If · is a norm, we can verify}
\CommentTok{\# that}
\NormalTok{Matrix}\SpecialCharTok{::}\FunctionTok{as.array}\NormalTok{(norms)}
\end{Highlighting}
\end{Shaded}

\begin{verbatim}
## logical(0)
\end{verbatim}

\begin{Shaded}
\begin{Highlighting}[]
\CommentTok{\# is a metric by using the properties of a norm to establish the three }
\CommentTok{\# properties of a metric above (Exercise 2.7).}
\FunctionTok{c}\NormalTok{(norms, }\AttributeTok{sigSlots =} \FunctionTok{c}\NormalTok{(}\StringTok{"target"}\NormalTok{, }\StringTok{"defined"}\NormalTok{))}
\end{Highlighting}
\end{Shaded}

\begin{verbatim}
## sigSlots1 sigSlots2 
##  "target" "defined"
\end{verbatim}

\begin{Shaded}
\begin{Highlighting}[]
\CommentTok{\# The general inner products, norms, and metrics defined above are relevant}
\CommentTok{\# in a wide range of applications. The sets on which they are defined can consist}
\CommentTok{\# of various types of objects. In the context of real vectors, the most common}
\CommentTok{\# inner product is the dot product; the most common norm is the Euclidean}
\CommentTok{\# norm that arises from the dot product; and the most common metric is the}
\CommentTok{\# one defined by the Euclidean norm, called the Euclidean distance.}
\FunctionTok{c}\NormalTok{(norms)}
\end{Highlighting}
\end{Shaded}

\begin{verbatim}
## logical(0)
\end{verbatim}

\begin{Shaded}
\begin{Highlighting}[]
\CommentTok{\# 2.1.8 Orthogonal Vectors and Orthogonal Vector Spaces}
\CommentTok{\# Two vectors v 1 and v 2 such that}
\CommentTok{\# v 1 , v 2 = 0}
\CommentTok{\# (2.23)}
\CommentTok{\# are said to be orthogonal, and this condition is denoted by v 1 ⊥ v 2 . }
\CommentTok{\# (Some{-}}
\CommentTok{\# times we exclude the zero vector from this definition, but it is not important}
\NormalTok{orthogonal }\OtherTok{\textless{}{-}} \FunctionTok{c}\NormalTok{(}\AttributeTok{v =} \DecValTok{1}\NormalTok{, }\AttributeTok{v =} \DecValTok{2}\NormalTok{, }\AttributeTok{v =} \DecValTok{3}\NormalTok{)}
\NormalTok{orthogonal}
\end{Highlighting}
\end{Shaded}

\begin{verbatim}
## v v v 
## 1 2 3
\end{verbatim}

\begin{Shaded}
\begin{Highlighting}[]
\CommentTok{\# 2.1 Operations on Vectors}
\CommentTok{\# 23}
\CommentTok{\# to do so.) Normalized vectors that are all orthogonal to each other are called}
\CommentTok{\# orthonormal vectors. (If the elements of the vectors are from the field of com{-}}
\CommentTok{\# plex numbers, orthogonality and normality are defined in terms of the dot}
\CommentTok{\# products of a vector with a sorted conjugate of a vector.)}
\FunctionTok{options}\NormalTok{(}\StringTok{"orthogonal"}\NormalTok{)}
\end{Highlighting}
\end{Shaded}

\begin{verbatim}
## $orthogonal
## NULL
\end{verbatim}

\begin{Shaded}
\begin{Highlighting}[]
\CommentTok{\# A set of nonzero vectors that are mutually orthogonal are necessarily lin{-}}
\CommentTok{\# early independent. To see this, we show it for any two orthogonal vectors}
\CommentTok{\# and then indicate the pattern that extends to three or more vectors. Sup{-}}
\CommentTok{\# pose v 1 and v 2 are nonzero and are orthogonal; that is, v 1 , v 2 = 0. We }
\CommentTok{\# see}
\CommentTok{\# immediately that if there is a scalar a such that v 1 = av 2 , then a must be}
\CommentTok{\# nonzero and we have a contradiction because v 1 , v 2 = a v 1 , v 1 = 0. For }
\CommentTok{\# three}
\CommentTok{\# mutually orthogonal vectors, v 1 , v 2 , and v 3 , we consider }
\CommentTok{\# v 1 = av 2 + bv 3 for a}
 \CommentTok{\#or b nonzero, and arrive at the same contradiction.}
\FunctionTok{set.seed}\NormalTok{(}\FunctionTok{c}\NormalTok{(}\AttributeTok{v =} \DecValTok{1}\NormalTok{, }\AttributeTok{v =} \DecValTok{2}\NormalTok{, }\AttributeTok{v =} \DecValTok{3}\NormalTok{))}
\NormalTok{orthogonal }\SpecialCharTok{+} \FunctionTok{sin}\NormalTok{(}\DecValTok{2}\NormalTok{)}
\end{Highlighting}
\end{Shaded}

\begin{verbatim}
##        v        v        v 
## 1.909297 2.909297 3.909297
\end{verbatim}

\begin{Shaded}
\begin{Highlighting}[]
\CommentTok{\# Two vector spaces V 1 and V 2 are said to be orthogonal, written V 1 ⊥ V 2 ,}
\CommentTok{\# if each vector in one is orthogonal to every vector in the other. }
\CommentTok{\# If V 1 ⊥ V 2 and}
\CommentTok{\# V 1 ⊕ V 2 = IR n , then V 2 is called the orthogonal complement of V 1 , }
\CommentTok{\# and this is}
\CommentTok{\# written as V 2 = V 1 ⊥ . More generally, if V 1 ⊥ V 2 and V 1 ⊕ V 2 = V, t}
\CommentTok{\# hen V 2 is}
\CommentTok{\# called the orthogonal complement of V 1 with respect to V. This is obviously}
\CommentTok{\# a symmetric relationship; if V 2 is the orthogonal complement of V 1 , then }
\CommentTok{\# V 1}
\CommentTok{\# is the orthogonal complement of V 2 .}
\NormalTok{vec }\OtherTok{\textless{}{-}} \FunctionTok{c}\NormalTok{(orthogonal, norms)}
\NormalTok{vec }
\end{Highlighting}
\end{Shaded}

\begin{verbatim}
## v v v 
## 1 2 3
\end{verbatim}

\begin{Shaded}
\begin{Highlighting}[]
\CommentTok{\# If B 1 is a basis set for V 1 , B 2 is a basis set for V 2 , and V 2 is the }
\CommentTok{\# orthogonal}
\CommentTok{\# complement of V 1 with respect to V, then B 1 ∪ B 2 is a basis set for V. }
\CommentTok{\# It is}
\CommentTok{\# a basis set because since V 1 and V 2 are orthogonal, it must be the case that}
\CommentTok{\# B 1 ∩ B 2 = ∅.}
\CommentTok{\# If V 1 ⊂ V, V 2 ⊂ V, V 1 ⊥ V 2 , and dim(V 1 ) + dim(V 2 ) = dim(V), then}
\CommentTok{\# V 1 ⊕ V 2 = V;}
\FunctionTok{dim}\NormalTok{(norms) }
\end{Highlighting}
\end{Shaded}

\begin{verbatim}
## NULL
\end{verbatim}

\begin{Shaded}
\begin{Highlighting}[]
\CommentTok{\# that is, V 2 is the orthogonal complement of V 1 . We see this by first letting }
\CommentTok{\# B 1 and B 2 be bases for V 1 and V 2 . Now V 1 ⊥ V 2 implies that B 1 ∩ B }
\CommentTok{\# 2 = ∅ and dim(V 1 ) + dim(V 2 ) = dim(V) implies \#(B 1 ) + \#(B 2 ) = \#(B), }
\CommentTok{\# for any basis set B for V; hence, B 1 ∪ B 2 is a basis set for V.}
\NormalTok{vb }\OtherTok{\textless{}{-}} \FunctionTok{c}\NormalTok{(}\AttributeTok{b =} \DecValTok{1}\NormalTok{, }\AttributeTok{b =} \DecValTok{2}\NormalTok{, }\AttributeTok{v =} \DecValTok{1}\NormalTok{, }\AttributeTok{v =} \DecValTok{2}\NormalTok{)}
\FunctionTok{dim}\NormalTok{(vb)}
\end{Highlighting}
\end{Shaded}

\begin{verbatim}
## NULL
\end{verbatim}

\begin{Shaded}
\begin{Highlighting}[]
\NormalTok{vb}
\end{Highlighting}
\end{Shaded}

\begin{verbatim}
## b b v v 
## 1 2 1 2
\end{verbatim}

\begin{Shaded}
\begin{Highlighting}[]
\CommentTok{\# 2.1.9 The “One Vector”}
\CommentTok{\# Another often useful vector is the vector with all elements equal to 1. We }
\CommentTok{\# call this the “one vector” and denote it by 1 or by 1 n . The one vector can }
\CommentTok{\# be used in the representation of the sum of the elements in a vector:}
\FunctionTok{c}\NormalTok{(}\StringTok{"https://tidyselect.r{-}lib.org/reference/faq{-}selection{-}context.html"}\NormalTok{, }
       \AttributeTok{vars =} \ConstantTok{NULL}\NormalTok{)}
\end{Highlighting}
\end{Shaded}

\begin{verbatim}
##                                                                     
## "https://tidyselect.r-lib.org/reference/faq-selection-context.html"
\end{verbatim}

\begin{Shaded}
\begin{Highlighting}[]
\CommentTok{\# The one vector is also called the “summing vector”.}
\FunctionTok{summary}\NormalTok{(}\AttributeTok{.data =} \StringTok{"dplyr"}\NormalTok{, vb, }\AttributeTok{groups =} \ConstantTok{NULL}\NormalTok{)}
\end{Highlighting}
\end{Shaded}

\begin{verbatim}
##    Min. 1st Qu.  Median    Mean 3rd Qu.    Max. 
##     1.0     1.0     1.5     1.5     2.0     2.0
\end{verbatim}

\end{document}
