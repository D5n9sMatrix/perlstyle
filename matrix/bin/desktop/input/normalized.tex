% Options for packages loaded elsewhere
\PassOptionsToPackage{unicode}{hyperref}
\PassOptionsToPackage{hyphens}{url}
%
\documentclass[
]{article}
\usepackage{amsmath,amssymb}
\usepackage{lmodern}
\usepackage{ifxetex,ifluatex}
\ifnum 0\ifxetex 1\fi\ifluatex 1\fi=0 % if pdftex
  \usepackage[T1]{fontenc}
  \usepackage[utf8]{inputenc}
  \usepackage{textcomp} % provide euro and other symbols
\else % if luatex or xetex
  \usepackage{unicode-math}
  \defaultfontfeatures{Scale=MatchLowercase}
  \defaultfontfeatures[\rmfamily]{Ligatures=TeX,Scale=1}
\fi
% Use upquote if available, for straight quotes in verbatim environments
\IfFileExists{upquote.sty}{\usepackage{upquote}}{}
\IfFileExists{microtype.sty}{% use microtype if available
  \usepackage[]{microtype}
  \UseMicrotypeSet[protrusion]{basicmath} % disable protrusion for tt fonts
}{}
\makeatletter
\@ifundefined{KOMAClassName}{% if non-KOMA class
  \IfFileExists{parskip.sty}{%
    \usepackage{parskip}
  }{% else
    \setlength{\parindent}{0pt}
    \setlength{\parskip}{6pt plus 2pt minus 1pt}}
}{% if KOMA class
  \KOMAoptions{parskip=half}}
\makeatother
\usepackage{xcolor}
\IfFileExists{xurl.sty}{\usepackage{xurl}}{} % add URL line breaks if available
\IfFileExists{bookmark.sty}{\usepackage{bookmark}}{\usepackage{hyperref}}
\hypersetup{
  pdftitle={normalized.r},
  pdfauthor={denis},
  hidelinks,
  pdfcreator={LaTeX via pandoc}}
\urlstyle{same} % disable monospaced font for URLs
\usepackage[margin=1in]{geometry}
\usepackage{color}
\usepackage{fancyvrb}
\newcommand{\VerbBar}{|}
\newcommand{\VERB}{\Verb[commandchars=\\\{\}]}
\DefineVerbatimEnvironment{Highlighting}{Verbatim}{commandchars=\\\{\}}
% Add ',fontsize=\small' for more characters per line
\usepackage{framed}
\definecolor{shadecolor}{RGB}{248,248,248}
\newenvironment{Shaded}{\begin{snugshade}}{\end{snugshade}}
\newcommand{\AlertTok}[1]{\textcolor[rgb]{0.94,0.16,0.16}{#1}}
\newcommand{\AnnotationTok}[1]{\textcolor[rgb]{0.56,0.35,0.01}{\textbf{\textit{#1}}}}
\newcommand{\AttributeTok}[1]{\textcolor[rgb]{0.77,0.63,0.00}{#1}}
\newcommand{\BaseNTok}[1]{\textcolor[rgb]{0.00,0.00,0.81}{#1}}
\newcommand{\BuiltInTok}[1]{#1}
\newcommand{\CharTok}[1]{\textcolor[rgb]{0.31,0.60,0.02}{#1}}
\newcommand{\CommentTok}[1]{\textcolor[rgb]{0.56,0.35,0.01}{\textit{#1}}}
\newcommand{\CommentVarTok}[1]{\textcolor[rgb]{0.56,0.35,0.01}{\textbf{\textit{#1}}}}
\newcommand{\ConstantTok}[1]{\textcolor[rgb]{0.00,0.00,0.00}{#1}}
\newcommand{\ControlFlowTok}[1]{\textcolor[rgb]{0.13,0.29,0.53}{\textbf{#1}}}
\newcommand{\DataTypeTok}[1]{\textcolor[rgb]{0.13,0.29,0.53}{#1}}
\newcommand{\DecValTok}[1]{\textcolor[rgb]{0.00,0.00,0.81}{#1}}
\newcommand{\DocumentationTok}[1]{\textcolor[rgb]{0.56,0.35,0.01}{\textbf{\textit{#1}}}}
\newcommand{\ErrorTok}[1]{\textcolor[rgb]{0.64,0.00,0.00}{\textbf{#1}}}
\newcommand{\ExtensionTok}[1]{#1}
\newcommand{\FloatTok}[1]{\textcolor[rgb]{0.00,0.00,0.81}{#1}}
\newcommand{\FunctionTok}[1]{\textcolor[rgb]{0.00,0.00,0.00}{#1}}
\newcommand{\ImportTok}[1]{#1}
\newcommand{\InformationTok}[1]{\textcolor[rgb]{0.56,0.35,0.01}{\textbf{\textit{#1}}}}
\newcommand{\KeywordTok}[1]{\textcolor[rgb]{0.13,0.29,0.53}{\textbf{#1}}}
\newcommand{\NormalTok}[1]{#1}
\newcommand{\OperatorTok}[1]{\textcolor[rgb]{0.81,0.36,0.00}{\textbf{#1}}}
\newcommand{\OtherTok}[1]{\textcolor[rgb]{0.56,0.35,0.01}{#1}}
\newcommand{\PreprocessorTok}[1]{\textcolor[rgb]{0.56,0.35,0.01}{\textit{#1}}}
\newcommand{\RegionMarkerTok}[1]{#1}
\newcommand{\SpecialCharTok}[1]{\textcolor[rgb]{0.00,0.00,0.00}{#1}}
\newcommand{\SpecialStringTok}[1]{\textcolor[rgb]{0.31,0.60,0.02}{#1}}
\newcommand{\StringTok}[1]{\textcolor[rgb]{0.31,0.60,0.02}{#1}}
\newcommand{\VariableTok}[1]{\textcolor[rgb]{0.00,0.00,0.00}{#1}}
\newcommand{\VerbatimStringTok}[1]{\textcolor[rgb]{0.31,0.60,0.02}{#1}}
\newcommand{\WarningTok}[1]{\textcolor[rgb]{0.56,0.35,0.01}{\textbf{\textit{#1}}}}
\usepackage{graphicx}
\makeatletter
\def\maxwidth{\ifdim\Gin@nat@width>\linewidth\linewidth\else\Gin@nat@width\fi}
\def\maxheight{\ifdim\Gin@nat@height>\textheight\textheight\else\Gin@nat@height\fi}
\makeatother
% Scale images if necessary, so that they will not overflow the page
% margins by default, and it is still possible to overwrite the defaults
% using explicit options in \includegraphics[width, height, ...]{}
\setkeys{Gin}{width=\maxwidth,height=\maxheight,keepaspectratio}
% Set default figure placement to htbp
\makeatletter
\def\fps@figure{htbp}
\makeatother
\setlength{\emergencystretch}{3em} % prevent overfull lines
\providecommand{\tightlist}{%
  \setlength{\itemsep}{0pt}\setlength{\parskip}{0pt}}
\setcounter{secnumdepth}{-\maxdimen} % remove section numbering
\ifluatex
  \usepackage{selnolig}  % disable illegal ligatures
\fi

\title{normalized.r}
\author{denis}
\date{2021-07-13}

\begin{document}
\maketitle

\begin{Shaded}
\begin{Highlighting}[]
\CommentTok{\#!/usr/bin/r}

\NormalTok{tibble}\SpecialCharTok{::}\FunctionTok{tibble}\NormalTok{(}\StringTok{"chelp"}\NormalTok{, }\FunctionTok{c}\NormalTok{(}\DecValTok{0}\SpecialCharTok{:}\DecValTok{1}\NormalTok{, }\AttributeTok{as.factor =} \ConstantTok{FALSE}\NormalTok{), }
               \AttributeTok{.name\_repair =} \FunctionTok{c}\NormalTok{(}\StringTok{"check\_unique"}\NormalTok{, }\StringTok{"universal"}\NormalTok{, }\StringTok{"minimal"}\NormalTok{))}
\end{Highlighting}
\end{Shaded}

\begin{verbatim}
## # A tibble: 3 x 2
##   `"chelp"` `c(0:1, as.factor = FALSE)`
##   <chr>                           <int>
## 1 chelp                               0
## 2 chelp                               1
## 3 chelp                               0
\end{verbatim}

\begin{Shaded}
\begin{Highlighting}[]
\CommentTok{\# Now, to establish a lower bound for x a , let us define a subset C of the}
\CommentTok{\# linear space consisting of all vectors (u 1 , . . . , u k ) such that}
\CommentTok{\# |u i | 2 = 1. This}
\NormalTok{ui }\OtherTok{\textless{}{-}} \DecValTok{2} \SpecialCharTok{+} \DecValTok{2}
\ControlFlowTok{for}\NormalTok{ (ui }\ControlFlowTok{in}  \DecValTok{2} \SpecialCharTok{+} \DecValTok{11}\NormalTok{) \{}
     \FunctionTok{c}\NormalTok{(ui)}
\NormalTok{\}}
\NormalTok{ui}
\end{Highlighting}
\end{Shaded}

\begin{verbatim}
## [1] 13
\end{verbatim}

\begin{Shaded}
\begin{Highlighting}[]
\CommentTok{\# 2 Vectors and Vector Spaces}
\CommentTok{\# set is obviously closed. Next, we define a function f (·) over this closed }
\CommentTok{\# subset by}
\NormalTok{f }\OtherTok{\textless{}{-}} \FunctionTok{c}\NormalTok{(}\FunctionTok{diag}\NormalTok{(}\DecValTok{0}\SpecialCharTok{:}\DecValTok{1}\NormalTok{, }\AttributeTok{nrow =} \DecValTok{1}\NormalTok{, }\AttributeTok{ncol =} \DecValTok{1}\NormalTok{, }\AttributeTok{names =} \ConstantTok{TRUE}\NormalTok{))}
\NormalTok{f}
\end{Highlighting}
\end{Shaded}

\begin{verbatim}
## [1] 0
\end{verbatim}

\begin{Shaded}
\begin{Highlighting}[]
\CommentTok{\# Because f is continuous, it attains a minimum in this closed}
\CommentTok{\# subset, say for}
\CommentTok{\# the vector u ∗ ; that is, f (u ∗ ) ≤ f (u) for any u such that}
\CommentTok{\# |u i | 2 = 1. Let}
\FunctionTok{c}\NormalTok{(}\DecValTok{0}\NormalTok{)}
\end{Highlighting}
\end{Shaded}

\begin{verbatim}
## [1] 0
\end{verbatim}

\begin{Shaded}
\begin{Highlighting}[]
\CommentTok{\# which must be positive, and again consider any x in the formed linear space}
\CommentTok{\# and express it in terms of the basis, x = c 1 v 1 + · · · c k v k . If x = 0, }
\CommentTok{\# we have}
\NormalTok{base}\SpecialCharTok{::}\FunctionTok{addNA}\NormalTok{(f, }\AttributeTok{ifany =} \ConstantTok{FALSE}\NormalTok{)}
\end{Highlighting}
\end{Shaded}

\begin{verbatim}
## [1] 0
## Levels: 0 <NA>
\end{verbatim}

\begin{Shaded}
\begin{Highlighting}[]
\CommentTok{\# where c̃ = (c 1 , · · · , c k )/( 1=i c 2 i ) 1/2 . Because cc is in the se}
\CommentTok{\# t C, f (c̃) ≥ r;}
\CommentTok{\# hence, combining this with the inequality above, we have}
\FunctionTok{c}\NormalTok{(}\FunctionTok{c}\NormalTok{(}\DecValTok{1}\NormalTok{,}\DecValTok{1}\NormalTok{,}\DecValTok{1}\NormalTok{,(}\DecValTok{5}\NormalTok{)}\SpecialCharTok{/}\NormalTok{(}\DecValTok{1}\SpecialCharTok{+} \DecValTok{1} \SpecialCharTok{+}\NormalTok{ (}\DecValTok{2} \SpecialCharTok{*} \DecValTok{1}\NormalTok{) }\SpecialCharTok{+} \DecValTok{1} \SpecialCharTok{/} \DecValTok{2}\NormalTok{)))}
\end{Highlighting}
\end{Shaded}

\begin{verbatim}
## [1] 1.000000 1.000000 1.000000 1.111111
\end{verbatim}

\begin{Shaded}
\begin{Highlighting}[]
\CommentTok{\# This expression holds for any norm · a and so, after obtaining similar bounds}
\CommentTok{\# for any other norm · b and then combining the inequalities for · a and · b ,}
\CommentTok{\# we have the bounds in the equivalence relation (2.17). (This is an equivalence}
\CommentTok{\# relation because it is reflexive, symmetric, and transitive. Its transitivity }
\CommentTok{\# is seen by the same argument that allowed us to go from the inequalities }
\CommentTok{\# involving ρ(·) to ones involving  ·  b .)}
\FunctionTok{c}\NormalTok{(}\AttributeTok{.data =} \StringTok{"dbplyr"}\NormalTok{, }\AttributeTok{.before =} \ConstantTok{NULL}\NormalTok{, }\AttributeTok{.after =} \ConstantTok{NULL}\NormalTok{)}
\end{Highlighting}
\end{Shaded}

\begin{verbatim}
##    .data 
## "dbplyr"
\end{verbatim}

\begin{Shaded}
\begin{Highlighting}[]
\NormalTok{p }\OtherTok{\textless{}{-}} \FunctionTok{list}\NormalTok{(}\StringTok{\textquotesingle{}./\textquotesingle{}}\NormalTok{)}
\NormalTok{b }\OtherTok{\textless{}{-}} \FunctionTok{c}\NormalTok{(}\FloatTok{2.17}\NormalTok{, }\StringTok{"keys{-}select"}\NormalTok{, }\DecValTok{1}\NormalTok{)}
\NormalTok{b}
\end{Highlighting}
\end{Shaded}

\begin{verbatim}
## [1] "2.17"        "keys-select" "1"
\end{verbatim}

\begin{Shaded}
\begin{Highlighting}[]
\NormalTok{p}
\end{Highlighting}
\end{Shaded}

\begin{verbatim}
## [[1]]
## [1] "./"
\end{verbatim}

\begin{Shaded}
\begin{Highlighting}[]
\CommentTok{\# Convergence of Sequences of Vectors}
\CommentTok{\# A sequence of real numbers a 1 , a 2 , . . . is said to converge to a kite }
\CommentTok{\# number a if for any given  \textgreater{} 0 there is an integer M such that, for k \textgreater{} M , }
\CommentTok{\# |a k − a| \textless{} , and we write lim k→∞ a k = a, or we write a k → a as k → ∞. I}
\CommentTok{\# f M does not depend on, the convergence is said to be uniform.}
\NormalTok{real }\OtherTok{\textless{}{-}} \FunctionTok{seq}\NormalTok{(}\FunctionTok{c}\NormalTok{(}\DecValTok{1}\NormalTok{,}\DecValTok{1}\NormalTok{,}\DecValTok{2}\NormalTok{,}\DecValTok{2}\NormalTok{,}\DecValTok{2}\NormalTok{) }\SpecialCharTok{\textgreater{}} \DecValTok{0}\NormalTok{)}
\ControlFlowTok{for}\NormalTok{ (real }\ControlFlowTok{in} \FunctionTok{c}\NormalTok{(}\DecValTok{1}\NormalTok{,}\DecValTok{2} \SpecialCharTok{{-}} \DecValTok{1} \SpecialCharTok{\textless{}} \DecValTok{2} \SpecialCharTok{+} \DecValTok{2} \SpecialCharTok{+} \DecValTok{1} \SpecialCharTok{+} \DecValTok{2} \SpecialCharTok{{-}} \DecValTok{1} \SpecialCharTok{+} \DecValTok{2}\NormalTok{)) \{}
    \FunctionTok{c}\NormalTok{(real)}
\NormalTok{\}}
\NormalTok{real }\SpecialCharTok{+} \FunctionTok{sin}\NormalTok{(real)}
\end{Highlighting}
\end{Shaded}

\begin{verbatim}
## [1] 1.841471
\end{verbatim}

\begin{Shaded}
\begin{Highlighting}[]
\CommentTok{\# We define convergence of a sequence of vectors in terms of the convergence}
\CommentTok{\# of a sequence of their norms, which is a sequence of real numbers. We say that}
\CommentTok{\# a sequence of vectors x 1 , x 2 , . . . (of the same order) converges to the }
\CommentTok{\# vector x with respect to the norm · if the sequence of real numbers x 1 − x, }
\CommentTok{\# x 2 − x, . . . converges to 0. Because of the bounds (2.17), the choice of }
\CommentTok{\# the norm is irrelevant, and so convergence of a sequence of vectors is }
\CommentTok{\# well{-}defined without reference to a specific norm. (This is the reason }
\CommentTok{\# equivalence of norms is an important property.)}
\NormalTok{norms }\OtherTok{\textless{}{-}} \FunctionTok{c}\NormalTok{(}\FunctionTok{c}\NormalTok{(}\DecValTok{1}\NormalTok{,}\DecValTok{1}\NormalTok{,}\DecValTok{1}\NormalTok{,}\DecValTok{2}\NormalTok{,}\DecValTok{2}\NormalTok{,}\DecValTok{2} \SpecialCharTok{+}\NormalTok{ (}\FloatTok{1.2} \SpecialCharTok{{-}} \FloatTok{11.1} \SpecialCharTok{+}\NormalTok{ (}\FloatTok{2.17}\NormalTok{))))}
\NormalTok{norms}
\end{Highlighting}
\end{Shaded}

\begin{verbatim}
## [1]  1.00  1.00  1.00  2.00  2.00 -5.73
\end{verbatim}

\begin{Shaded}
\begin{Highlighting}[]
\CommentTok{\# Norms Induced by Inner Products}
\CommentTok{\# There is a close relationship between a norm and an inner product. For any}
\CommentTok{\# inner product space with inner product ·, · , a norm of an element of the}
\CommentTok{\# space can be defined in terms of the square root of the inner product of the}
\CommentTok{\# element with itself:}
\NormalTok{inner }\OtherTok{\textless{}{-}}\NormalTok{ prod}
\NormalTok{inner}
\end{Highlighting}
\end{Shaded}

\begin{verbatim}
## function (..., na.rm = FALSE)  .Primitive("prod")
\end{verbatim}

\begin{Shaded}
\begin{Highlighting}[]
\CommentTok{\# Any function · defined in this way satisfies the properties of a norm. It is}
\CommentTok{\# easy to see that x satisfies the first two properties of a norm, nonnegativity}
\CommentTok{\# and scalar equivariance. Now, consider the square of the right{-}hand side of}
\CommentTok{\# the triangle inequality, x + y:}
\FunctionTok{trigamma}\NormalTok{(}\FunctionTok{c}\NormalTok{(}\DecValTok{1} \SpecialCharTok{+} \DecValTok{1}\NormalTok{))}
\end{Highlighting}
\end{Shaded}

\begin{verbatim}
## [1] 0.6449341
\end{verbatim}

\begin{Shaded}
\begin{Highlighting}[]
\CommentTok{\# hence, the triangle inequality holds. Therefore, given an inner product, }
\CommentTok{\# x, y , x, x is a norm.}
\FunctionTok{trigamma}\NormalTok{(}\FunctionTok{c}\NormalTok{(}\DecValTok{0}\NormalTok{, }\DecValTok{1}\NormalTok{,}\DecValTok{1}\NormalTok{,}\DecValTok{1}\NormalTok{))}
\end{Highlighting}
\end{Shaded}

\begin{verbatim}
## [1]      Inf 1.644934 1.644934 1.644934
\end{verbatim}

\begin{Shaded}
\begin{Highlighting}[]
\CommentTok{\# Equation (2.18) defines a norm given any inner product. It is called the}
\CommentTok{\# norm induced by the inner product. In the case of vectors and the inner }
\CommentTok{\# product we defined for vectors in equation (2.9), the induced norm is the }
\CommentTok{\# L 2 norm, · 2 , defined above.}
\NormalTok{eq }\OtherTok{\textless{}{-}}\NormalTok{ inner }
\NormalTok{l2 }\OtherTok{\textless{}{-}}\NormalTok{ norms }
\NormalTok{eq}
\end{Highlighting}
\end{Shaded}

\begin{verbatim}
## function (..., na.rm = FALSE)  .Primitive("prod")
\end{verbatim}

\begin{Shaded}
\begin{Highlighting}[]
\NormalTok{l2}
\end{Highlighting}
\end{Shaded}

\begin{verbatim}
## [1]  1.00  1.00  1.00  2.00  2.00 -5.73
\end{verbatim}

\begin{Shaded}
\begin{Highlighting}[]
\CommentTok{\# In the following, when we use the unqualified symbol · for a vector}
\CommentTok{\# norm, we will mean the L 2 norm; that is, the Euclidean norm, the induced}
\CommentTok{\# norm.}
\FunctionTok{c}\NormalTok{(l2, }\AttributeTok{y =} \ConstantTok{NULL}\NormalTok{, }\AttributeTok{circles =} \FunctionTok{c}\NormalTok{(}\DecValTok{2}\NormalTok{), }\AttributeTok{squares =} \FunctionTok{c}\NormalTok{(}\DecValTok{2}\NormalTok{), }\AttributeTok{rectangles =} \FunctionTok{c}\NormalTok{(}\DecValTok{4}\NormalTok{),}
        \AttributeTok{stars =} \ConstantTok{NA}\NormalTok{, }\AttributeTok{thermometers =} \FunctionTok{c}\NormalTok{(}\StringTok{"fg"}\NormalTok{), }\AttributeTok{boxplots =} \FunctionTok{c}\NormalTok{(}\DecValTok{2}\NormalTok{),}
        \AttributeTok{inches =} \ConstantTok{FALSE}\NormalTok{, }\AttributeTok{add =} \ConstantTok{TRUE}\NormalTok{, }\AttributeTok{fg =} \FunctionTok{par}\NormalTok{(}\StringTok{"col"}\NormalTok{), }\AttributeTok{bg =} \ConstantTok{NA}\NormalTok{,}
        \AttributeTok{xlab =} \ConstantTok{NULL}\NormalTok{, }\AttributeTok{ylab =} \ConstantTok{NULL}\NormalTok{, }\AttributeTok{main =} \ConstantTok{NULL}\NormalTok{, }\AttributeTok{xlim =} \ConstantTok{NULL}\NormalTok{, }\AttributeTok{ylim =} \ConstantTok{NULL}\NormalTok{)}
\end{Highlighting}
\end{Shaded}

\begin{verbatim}
##                                                                               
##          "1"          "1"          "1"          "2"          "2"      "-5.73" 
##      circles      squares   rectangles        stars thermometers     boxplots 
##          "2"          "2"          "4"           NA         "fg"          "2" 
##       inches          add           fg           bg 
##      "FALSE"       "TRUE"      "black"           NA
\end{verbatim}

\begin{Shaded}
\begin{Highlighting}[]
\CommentTok{\# 2.1.6 Normalized Vectors}
\CommentTok{\# The Euclidean norm of a vector corresponds to the length of the vector x in a}
\CommentTok{\# natural way; that is, it agrees with our intuition regarding “length”. }
\CommentTok{\# Although,}
\CommentTok{\# as we have seen, this is just one of many vector norms, in most applications}
\CommentTok{\# it is the most useful one. (I must warn you, however, that occasionally I will}
\CommentTok{\# carelessly but naturally use “length” to refer to the order of a vector; }
\CommentTok{\# that is,}
\CommentTok{\# the number of elements. This usage is common in computer software packages}
\CommentTok{\# such as R and SAS IML, and software necessarily shapes our vocabulary.)}
\NormalTok{natural }\OtherTok{\textless{}{-}} \FunctionTok{length}\NormalTok{(}\DecValTok{0}\SpecialCharTok{:}\DecValTok{999}\NormalTok{)}
\NormalTok{natural}
\end{Highlighting}
\end{Shaded}

\begin{verbatim}
## [1] 1000
\end{verbatim}

\begin{Shaded}
\begin{Highlighting}[]
\CommentTok{\# Dividing a given vector by its length normalizes the vector, and the re{-}}
\CommentTok{\# sulking vector with length 1 is said to be normalized; thus}
\FunctionTok{length}\NormalTok{(}\DecValTok{1}\NormalTok{)}
\end{Highlighting}
\end{Shaded}

\begin{verbatim}
## [1] 1
\end{verbatim}

\end{document}
