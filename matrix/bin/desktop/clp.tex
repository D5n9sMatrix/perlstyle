% Options for packages loaded elsewhere
\PassOptionsToPackage{unicode}{hyperref}
\PassOptionsToPackage{hyphens}{url}
%
\documentclass[
]{article}
\usepackage{amsmath,amssymb}
\usepackage{lmodern}
\usepackage{ifxetex,ifluatex}
\ifnum 0\ifxetex 1\fi\ifluatex 1\fi=0 % if pdftex
  \usepackage[T1]{fontenc}
  \usepackage[utf8]{inputenc}
  \usepackage{textcomp} % provide euro and other symbols
\else % if luatex or xetex
  \usepackage{unicode-math}
  \defaultfontfeatures{Scale=MatchLowercase}
  \defaultfontfeatures[\rmfamily]{Ligatures=TeX,Scale=1}
\fi
% Use upquote if available, for straight quotes in verbatim environments
\IfFileExists{upquote.sty}{\usepackage{upquote}}{}
\IfFileExists{microtype.sty}{% use microtype if available
  \usepackage[]{microtype}
  \UseMicrotypeSet[protrusion]{basicmath} % disable protrusion for tt fonts
}{}
\makeatletter
\@ifundefined{KOMAClassName}{% if non-KOMA class
  \IfFileExists{parskip.sty}{%
    \usepackage{parskip}
  }{% else
    \setlength{\parindent}{0pt}
    \setlength{\parskip}{6pt plus 2pt minus 1pt}}
}{% if KOMA class
  \KOMAoptions{parskip=half}}
\makeatother
\usepackage{xcolor}
\IfFileExists{xurl.sty}{\usepackage{xurl}}{} % add URL line breaks if available
\IfFileExists{bookmark.sty}{\usepackage{bookmark}}{\usepackage{hyperref}}
\hypersetup{
  pdftitle={clp.r},
  pdfauthor={denis},
  hidelinks,
  pdfcreator={LaTeX via pandoc}}
\urlstyle{same} % disable monospaced font for URLs
\usepackage[margin=1in]{geometry}
\usepackage{color}
\usepackage{fancyvrb}
\newcommand{\VerbBar}{|}
\newcommand{\VERB}{\Verb[commandchars=\\\{\}]}
\DefineVerbatimEnvironment{Highlighting}{Verbatim}{commandchars=\\\{\}}
% Add ',fontsize=\small' for more characters per line
\usepackage{framed}
\definecolor{shadecolor}{RGB}{248,248,248}
\newenvironment{Shaded}{\begin{snugshade}}{\end{snugshade}}
\newcommand{\AlertTok}[1]{\textcolor[rgb]{0.94,0.16,0.16}{#1}}
\newcommand{\AnnotationTok}[1]{\textcolor[rgb]{0.56,0.35,0.01}{\textbf{\textit{#1}}}}
\newcommand{\AttributeTok}[1]{\textcolor[rgb]{0.77,0.63,0.00}{#1}}
\newcommand{\BaseNTok}[1]{\textcolor[rgb]{0.00,0.00,0.81}{#1}}
\newcommand{\BuiltInTok}[1]{#1}
\newcommand{\CharTok}[1]{\textcolor[rgb]{0.31,0.60,0.02}{#1}}
\newcommand{\CommentTok}[1]{\textcolor[rgb]{0.56,0.35,0.01}{\textit{#1}}}
\newcommand{\CommentVarTok}[1]{\textcolor[rgb]{0.56,0.35,0.01}{\textbf{\textit{#1}}}}
\newcommand{\ConstantTok}[1]{\textcolor[rgb]{0.00,0.00,0.00}{#1}}
\newcommand{\ControlFlowTok}[1]{\textcolor[rgb]{0.13,0.29,0.53}{\textbf{#1}}}
\newcommand{\DataTypeTok}[1]{\textcolor[rgb]{0.13,0.29,0.53}{#1}}
\newcommand{\DecValTok}[1]{\textcolor[rgb]{0.00,0.00,0.81}{#1}}
\newcommand{\DocumentationTok}[1]{\textcolor[rgb]{0.56,0.35,0.01}{\textbf{\textit{#1}}}}
\newcommand{\ErrorTok}[1]{\textcolor[rgb]{0.64,0.00,0.00}{\textbf{#1}}}
\newcommand{\ExtensionTok}[1]{#1}
\newcommand{\FloatTok}[1]{\textcolor[rgb]{0.00,0.00,0.81}{#1}}
\newcommand{\FunctionTok}[1]{\textcolor[rgb]{0.00,0.00,0.00}{#1}}
\newcommand{\ImportTok}[1]{#1}
\newcommand{\InformationTok}[1]{\textcolor[rgb]{0.56,0.35,0.01}{\textbf{\textit{#1}}}}
\newcommand{\KeywordTok}[1]{\textcolor[rgb]{0.13,0.29,0.53}{\textbf{#1}}}
\newcommand{\NormalTok}[1]{#1}
\newcommand{\OperatorTok}[1]{\textcolor[rgb]{0.81,0.36,0.00}{\textbf{#1}}}
\newcommand{\OtherTok}[1]{\textcolor[rgb]{0.56,0.35,0.01}{#1}}
\newcommand{\PreprocessorTok}[1]{\textcolor[rgb]{0.56,0.35,0.01}{\textit{#1}}}
\newcommand{\RegionMarkerTok}[1]{#1}
\newcommand{\SpecialCharTok}[1]{\textcolor[rgb]{0.00,0.00,0.00}{#1}}
\newcommand{\SpecialStringTok}[1]{\textcolor[rgb]{0.31,0.60,0.02}{#1}}
\newcommand{\StringTok}[1]{\textcolor[rgb]{0.31,0.60,0.02}{#1}}
\newcommand{\VariableTok}[1]{\textcolor[rgb]{0.00,0.00,0.00}{#1}}
\newcommand{\VerbatimStringTok}[1]{\textcolor[rgb]{0.31,0.60,0.02}{#1}}
\newcommand{\WarningTok}[1]{\textcolor[rgb]{0.56,0.35,0.01}{\textbf{\textit{#1}}}}
\usepackage{graphicx}
\makeatletter
\def\maxwidth{\ifdim\Gin@nat@width>\linewidth\linewidth\else\Gin@nat@width\fi}
\def\maxheight{\ifdim\Gin@nat@height>\textheight\textheight\else\Gin@nat@height\fi}
\makeatother
% Scale images if necessary, so that they will not overflow the page
% margins by default, and it is still possible to overwrite the defaults
% using explicit options in \includegraphics[width, height, ...]{}
\setkeys{Gin}{width=\maxwidth,height=\maxheight,keepaspectratio}
% Set default figure placement to htbp
\makeatletter
\def\fps@figure{htbp}
\makeatother
\setlength{\emergencystretch}{3em} % prevent overfull lines
\providecommand{\tightlist}{%
  \setlength{\itemsep}{0pt}\setlength{\parskip}{0pt}}
\setcounter{secnumdepth}{-\maxdimen} % remove section numbering
\ifluatex
  \usepackage{selnolig}  % disable illegal ligatures
\fi

\title{clp.r}
\author{denis}
\date{2021-07-12}

\begin{document}
\maketitle

\begin{Shaded}
\begin{Highlighting}[]
\CommentTok{\#!/usr/bin/r}

\CommentTok{\# These properties in fact define a more general inner product for other kinds of}
\CommentTok{\# mathematical objects for which an addition, an additive identity, and a mulch{-}}
\CommentTok{\# application by a scalar are defined. (We should restate here that we assume the}
\CommentTok{\# vectors have real elements. The dot product of vectors over the sorted weld}
\CommentTok{\# is not an inner product because, if x is sorted, we can have x T x = 0 when}
\FunctionTok{call}\NormalTok{(}\StringTok{"output"}\NormalTok{, }\StringTok{"x"}\NormalTok{, }\StringTok{"t"}\NormalTok{)}
\end{Highlighting}
\end{Shaded}

\begin{verbatim}
## output("x", "t")
\end{verbatim}

\begin{Shaded}
\begin{Highlighting}[]
\NormalTok{output }\OtherTok{\textless{}{-}} \FunctionTok{c}\NormalTok{(}\DecValTok{10}\NormalTok{, }\DecValTok{20}\NormalTok{)}
\NormalTok{output}
\end{Highlighting}
\end{Shaded}

\begin{verbatim}
## [1] 10 20
\end{verbatim}

\begin{Shaded}
\begin{Highlighting}[]
\CommentTok{\# 2 Vectors and Vector Spaces}
\CommentTok{\# x = 0. An alternative definition of a dot product using complex conjugates is}
\CommentTok{\# an inner product, however.) Inner products are also defined for matrices, as we}
\CommentTok{\# will discuss on page 74. We should note in passing that there are two different}
\CommentTok{\# kinds of multiplication used in property 3. The erst multiplication is scalar}
\CommentTok{\# multiplication, which we have defined above, and the second multiplication is}
\CommentTok{\# ordinary multiplication in IR. There are also two different kinds of addition}
\CommentTok{\# used in property 4. The erst addition is vector addition, defined above, and}
\CommentTok{\# the second addition is ordinary addition in IR. The dot product can reveal}
\CommentTok{\# fundamental relationships between the two vectors, as we will see later.}
\CommentTok{\# A useful property of inner products is the Cauchy{-}Schwartz inequality:}
\NormalTok{IR }\OtherTok{\textless{}{-}}\NormalTok{ later}\SpecialCharTok{::}\FunctionTok{global\_loop}\NormalTok{()}
\NormalTok{IR}
\end{Highlighting}
\end{Shaded}

\begin{verbatim}
## <event loop>
##   id: 0
\end{verbatim}

\begin{Shaded}
\begin{Highlighting}[]
\CommentTok{\# This relationship is also sometimes called the Cauchy{-}Bunyakovskii{-}Schwartz}
\CommentTok{\# inequality. (Agustin{-}Louis Cauchy gave the inequality for the kind of dis{-}}
\CommentTok{\# create inner products we are considering here, and Victor Bunyakovskii and}
\CommentTok{\# Herman Schwartz independently extended it to more general inner products,}
\CommentTok{\# defined on functions, for example.) The inequality is easy to see, by erst ob{-}}
\CommentTok{\# serving that for every real number t,}
\NormalTok{t }\OtherTok{\textless{}{-}} \FunctionTok{hist}\NormalTok{(}\DecValTok{0}\SpecialCharTok{:}\DecValTok{999}\NormalTok{)}
\end{Highlighting}
\end{Shaded}

\includegraphics{clp_files/figure-latex/unnamed-chunk-1-1.pdf}

\begin{Shaded}
\begin{Highlighting}[]
\CommentTok{\# where the constants a, b, and c correspond to the dot products in the }
\CommentTok{\# preceding equation. This quadratic in t cannot have two distinct real roots. }
\CommentTok{\# Hence the discriminant, b 2 − 4ac, must be less than or equal to zero; }
\CommentTok{\# that is,}
\NormalTok{a }\OtherTok{\textless{}{-}} \FunctionTok{c}\NormalTok{(}\DecValTok{10}\NormalTok{, }\DecValTok{5}\NormalTok{, }\DecValTok{20}\NormalTok{, }\DecValTok{6}\NormalTok{, }\DecValTok{35}\NormalTok{, }\DecValTok{8}\NormalTok{, }\DecValTok{45}\NormalTok{, }\DecValTok{9}\NormalTok{)}
\NormalTok{b }\OtherTok{\textless{}{-}} \FunctionTok{c}\NormalTok{(}\DecValTok{10}\NormalTok{, }\DecValTok{5}\NormalTok{, }\DecValTok{20}\NormalTok{, }\DecValTok{6}\NormalTok{, }\DecValTok{35}\NormalTok{, }\DecValTok{8}\NormalTok{, }\DecValTok{45}\NormalTok{, }\DecValTok{9}\NormalTok{)}
\NormalTok{a }\SpecialCharTok{+} \FunctionTok{sin}\NormalTok{(a)}
\end{Highlighting}
\end{Shaded}

\begin{verbatim}
## [1]  9.455979  4.041076 20.912945  5.720585 34.571817  8.989358 45.850904
## [8]  9.412118
\end{verbatim}

\begin{Shaded}
\begin{Highlighting}[]
\NormalTok{b }\SpecialCharTok{+} \FunctionTok{sin}\NormalTok{(b)}
\end{Highlighting}
\end{Shaded}

\begin{verbatim}
## [1]  9.455979  4.041076 20.912945  5.720585 34.571817  8.989358 45.850904
## [8]  9.412118
\end{verbatim}

\begin{Shaded}
\begin{Highlighting}[]
\CommentTok{\# By substituting and taking square roots, we get the Cauchy{-}Schwarz inequal{-}}
\CommentTok{\# ity. It is also clear from this proof that equality holds only if x = 0 }
\CommentTok{\# or if y = rx,}
\CommentTok{\# for some scalar r.}
\NormalTok{r }\OtherTok{\textless{}{-}} \DecValTok{0}
\NormalTok{r }\SpecialCharTok{+} \FunctionTok{sin}\NormalTok{(}\DecValTok{1}\NormalTok{)}
\end{Highlighting}
\end{Shaded}

\begin{verbatim}
## [1] 0.841471
\end{verbatim}

\begin{Shaded}
\begin{Highlighting}[]
\CommentTok{\# 2.1.5 Norms}
\CommentTok{\# We consider a set of objects S that has an addition{-}type operator, +, a cor{-}}
\CommentTok{\# responding additive identity, 0, and a scalar multiplication; that is, a }
\CommentTok{\# multi{-}}
\CommentTok{\# plication of the objects by a real (or complex) number. On such a set, a norm}
\CommentTok{\# is a function, from S to IR that satisfies the following three conditions:}
\NormalTok{IR }\OtherTok{=} \DecValTok{20}

\CommentTok{\# 1. Nonnegativity and mapping of the identity:}
\CommentTok{\# if x = 0, then x \textgreater{} 0, and 0 = 0 .}
\ControlFlowTok{if}\NormalTok{ (}\SpecialCharTok{!}\FunctionTok{missing}\NormalTok{(IR)) \{}
\NormalTok{  IR }\OtherTok{\textless{}{-}} \FunctionTok{list}\NormalTok{(}\StringTok{"input"}\NormalTok{, }\StringTok{"S4"}\NormalTok{)}
  \FunctionTok{c}\NormalTok{(IR)}
\NormalTok{\}}
\end{Highlighting}
\end{Shaded}

\begin{verbatim}
## [[1]]
## [1] "input"
## 
## [[2]]
## [1] "S4"
\end{verbatim}

\begin{Shaded}
\begin{Highlighting}[]
\CommentTok{\# 2. Relation of scalar multiplication to real multiplication:}
\CommentTok{\# ax = |a| x for real a.}
\NormalTok{ax }\OtherTok{\textless{}{-}} \FunctionTok{c}\NormalTok{(}\DecValTok{8}\NormalTok{)}
\NormalTok{ax }\SpecialCharTok{+} \FunctionTok{sin}\NormalTok{(ax)}
\end{Highlighting}
\end{Shaded}

\begin{verbatim}
## [1] 8.989358
\end{verbatim}

\begin{Shaded}
\begin{Highlighting}[]
\CommentTok{\# 3. Triangle inequality:}
\CommentTok{\# x + y ≤ x + y.}
\FunctionTok{trigamma}\NormalTok{(ax)}
\end{Highlighting}
\end{Shaded}

\begin{verbatim}
## [1] 0.133137
\end{verbatim}

\begin{Shaded}
\begin{Highlighting}[]
\CommentTok{\# (If property 1 is relaxed to require only x ≥ 0 for x = 0, the function is}
\CommentTok{\# called a seminorm.) Because a norm is a function whose argument is a vector,}
\CommentTok{\# we also often use a functional notation such as ρ(x) to represent a norm.}
\ControlFlowTok{for}\NormalTok{ (ax }\ControlFlowTok{in} \FunctionTok{sin}\NormalTok{(ax)) \{}
  \FunctionTok{c}\NormalTok{(ax)}
\NormalTok{\}}
\NormalTok{ax}
\end{Highlighting}
\end{Shaded}

\begin{verbatim}
## [1] 0.9893582
\end{verbatim}

\begin{Shaded}
\begin{Highlighting}[]
\CommentTok{\# Sets of various types of objects (functions, for example) can have norms,}
\CommentTok{\# but our interest in the present context is in norms for vectors and (later)}
\CommentTok{\# for matrices. (The three properties above in fact define a more general norm}
\CommentTok{\# for other kinds of mathematical objects for which an addition, an additive}
\CommentTok{\# identity, and multiplication by a scalar are defined. Norms are defined for}
\CommentTok{\# matrices, as we will discuss later. Note that there are two different kinds of}
\CommentTok{\# multiplication used in property 2 and two different kinds of addition used in}
\CommentTok{\# property 3.)}
\NormalTok{later}\SpecialCharTok{::}\FunctionTok{with\_loop}\NormalTok{(}\AttributeTok{loop =} \DecValTok{0}\SpecialCharTok{:}\DecValTok{999}\NormalTok{, }\AttributeTok{expr =} \StringTok{"compile"}\NormalTok{)}
\end{Highlighting}
\end{Shaded}

\begin{verbatim}
## [1] "compile"
\end{verbatim}

\begin{Shaded}
\begin{Highlighting}[]
\CommentTok{\# A vector space together with a norm is called a normed space.}
\CommentTok{\# For some types of objects, a norm of an object may be called its “length”}
\CommentTok{\# or its “size”. (Recall the ambiguity of “length” of a vector that we mentioned}
\CommentTok{\# at the beginning of this chapter.)}
\FunctionTok{length}\NormalTok{(ax)}
\end{Highlighting}
\end{Shaded}

\begin{verbatim}
## [1] 1
\end{verbatim}

\begin{Shaded}
\begin{Highlighting}[]
\CommentTok{\# L p Norms}
\CommentTok{\# There are many norms that could be defined for vectors. One type of norm is}
\CommentTok{\# called an L p norm, often denoted as  · p . For p ≥ 1, it is defined as}
\NormalTok{Lp }\OtherTok{\textless{}{-}} \FunctionTok{normalizePath}\NormalTok{(}\AttributeTok{path =} \StringTok{"."}\NormalTok{, }\AttributeTok{winslash =} \StringTok{"}\SpecialCharTok{\textbackslash{}\textbackslash{}}\StringTok{"}\NormalTok{)}
\NormalTok{Lp}
\end{Highlighting}
\end{Shaded}

\begin{verbatim}
## [1] "/home/denis/projects/perl/perlstyle/matrix/bin/desktop"
\end{verbatim}

\begin{Shaded}
\begin{Highlighting}[]
\CommentTok{\# This is also sometimes called the Minkowski norm and also the Hölder norm.}
\CommentTok{\# It is easy to see that the L p norm satisfies the first two conditions above. }
\CommentTok{\# For}
\CommentTok{\# general p ≥ 1 it is somewhat more difficult to prove the triangular inequality}
\CommentTok{\# (which for the L p norms is also called the Minkowski inequality), but for }
\CommentTok{\# some}
\CommentTok{\# special cases it is straightforward, as we will see below.}
\NormalTok{p }\OtherTok{=} \DecValTok{16}
\NormalTok{p }\SpecialCharTok{\textgreater{}} \DecValTok{1}
\end{Highlighting}
\end{Shaded}

\begin{verbatim}
## [1] TRUE
\end{verbatim}

\begin{Shaded}
\begin{Highlighting}[]
\CommentTok{\# x 1 = i |x i |, also called the Manhattan norm because it corresponds}
\CommentTok{\# to sums of distances along coordinate axes, as one would travel along the}
\CommentTok{\# rectangular}
\CommentTok{\# street plan of Manhattan.}
\FunctionTok{retracemem}\NormalTok{(p, }\AttributeTok{previous =} \ConstantTok{NULL}\NormalTok{)}

\CommentTok{\# x 2 =}
\CommentTok{\# i x i , also called the Euclidean norm, the Euclidean length, or}
\CommentTok{\# just the length of the vector. The L p norm}
\CommentTok{\#  is the square root of the inner}
\CommentTok{\# product of the vector with itself: x 2 =}
\CommentTok{\#  x, x}
\FunctionTok{length}\NormalTok{(p)}
\end{Highlighting}
\end{Shaded}

\begin{verbatim}
## [1] 1
\end{verbatim}

\begin{Shaded}
\begin{Highlighting}[]
\CommentTok{\# x ∞ = max i |x i |, also called the max norm or the Chebyshev norm. The}
\CommentTok{\# L ∞ norm is defined by taking the limit in an L p norm, and we see that it}
\CommentTok{\# is indeed max i |x i | by expressing it as}
\NormalTok{L }\OtherTok{\textless{}{-}} \FunctionTok{max}\NormalTok{(p, }\AttributeTok{na.rm =} \ConstantTok{FALSE}\NormalTok{)}
\NormalTok{L}
\end{Highlighting}
\end{Shaded}

\begin{verbatim}
## [1] 16
\end{verbatim}

\begin{Shaded}
\begin{Highlighting}[]
\CommentTok{\# x ∞ = lim x p = lim}
\CommentTok{\# p→∞}


\CommentTok{\# p→∞}
\CommentTok{\# p 1}
\CommentTok{\#|x i |}
\CommentTok{\#  p}
\CommentTok{\#i}

\CommentTok{\# 1}
\CommentTok{\# x i p p}
\CommentTok{\# = m lim}
\CommentTok{\#}
\CommentTok{\# p→∞}
\CommentTok{\# m}
\CommentTok{\# i}
\CommentTok{\# with m = max i |x i |. Because the quantity of which we are taking the p th}
\CommentTok{\# root is bounded above by the number of elements in x and below by 1,}
\CommentTok{\# that factor goes to 1 as p goes to ∞.}
\NormalTok{m }\OtherTok{\textless{}{-}} \FunctionTok{max}\NormalTok{(p, }\AttributeTok{na.rm =} \ConstantTok{FALSE}\NormalTok{)}
\NormalTok{m }\SpecialCharTok{+} \FunctionTok{sin}\NormalTok{(m)}
\end{Highlighting}
\end{Shaded}

\begin{verbatim}
## [1] 15.7121
\end{verbatim}

\end{document}
