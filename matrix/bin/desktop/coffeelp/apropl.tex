% Options for packages loaded elsewhere
\PassOptionsToPackage{unicode}{hyperref}
\PassOptionsToPackage{hyphens}{url}
%
\documentclass[
]{article}
\usepackage{amsmath,amssymb}
\usepackage{lmodern}
\usepackage{ifxetex,ifluatex}
\ifnum 0\ifxetex 1\fi\ifluatex 1\fi=0 % if pdftex
  \usepackage[T1]{fontenc}
  \usepackage[utf8]{inputenc}
  \usepackage{textcomp} % provide euro and other symbols
\else % if luatex or xetex
  \usepackage{unicode-math}
  \defaultfontfeatures{Scale=MatchLowercase}
  \defaultfontfeatures[\rmfamily]{Ligatures=TeX,Scale=1}
\fi
% Use upquote if available, for straight quotes in verbatim environments
\IfFileExists{upquote.sty}{\usepackage{upquote}}{}
\IfFileExists{microtype.sty}{% use microtype if available
  \usepackage[]{microtype}
  \UseMicrotypeSet[protrusion]{basicmath} % disable protrusion for tt fonts
}{}
\makeatletter
\@ifundefined{KOMAClassName}{% if non-KOMA class
  \IfFileExists{parskip.sty}{%
    \usepackage{parskip}
  }{% else
    \setlength{\parindent}{0pt}
    \setlength{\parskip}{6pt plus 2pt minus 1pt}}
}{% if KOMA class
  \KOMAoptions{parskip=half}}
\makeatother
\usepackage{xcolor}
\IfFileExists{xurl.sty}{\usepackage{xurl}}{} % add URL line breaks if available
\IfFileExists{bookmark.sty}{\usepackage{bookmark}}{\usepackage{hyperref}}
\hypersetup{
  pdftitle={apropl.r},
  pdfauthor={denis},
  hidelinks,
  pdfcreator={LaTeX via pandoc}}
\urlstyle{same} % disable monospaced font for URLs
\usepackage[margin=1in]{geometry}
\usepackage{color}
\usepackage{fancyvrb}
\newcommand{\VerbBar}{|}
\newcommand{\VERB}{\Verb[commandchars=\\\{\}]}
\DefineVerbatimEnvironment{Highlighting}{Verbatim}{commandchars=\\\{\}}
% Add ',fontsize=\small' for more characters per line
\usepackage{framed}
\definecolor{shadecolor}{RGB}{248,248,248}
\newenvironment{Shaded}{\begin{snugshade}}{\end{snugshade}}
\newcommand{\AlertTok}[1]{\textcolor[rgb]{0.94,0.16,0.16}{#1}}
\newcommand{\AnnotationTok}[1]{\textcolor[rgb]{0.56,0.35,0.01}{\textbf{\textit{#1}}}}
\newcommand{\AttributeTok}[1]{\textcolor[rgb]{0.77,0.63,0.00}{#1}}
\newcommand{\BaseNTok}[1]{\textcolor[rgb]{0.00,0.00,0.81}{#1}}
\newcommand{\BuiltInTok}[1]{#1}
\newcommand{\CharTok}[1]{\textcolor[rgb]{0.31,0.60,0.02}{#1}}
\newcommand{\CommentTok}[1]{\textcolor[rgb]{0.56,0.35,0.01}{\textit{#1}}}
\newcommand{\CommentVarTok}[1]{\textcolor[rgb]{0.56,0.35,0.01}{\textbf{\textit{#1}}}}
\newcommand{\ConstantTok}[1]{\textcolor[rgb]{0.00,0.00,0.00}{#1}}
\newcommand{\ControlFlowTok}[1]{\textcolor[rgb]{0.13,0.29,0.53}{\textbf{#1}}}
\newcommand{\DataTypeTok}[1]{\textcolor[rgb]{0.13,0.29,0.53}{#1}}
\newcommand{\DecValTok}[1]{\textcolor[rgb]{0.00,0.00,0.81}{#1}}
\newcommand{\DocumentationTok}[1]{\textcolor[rgb]{0.56,0.35,0.01}{\textbf{\textit{#1}}}}
\newcommand{\ErrorTok}[1]{\textcolor[rgb]{0.64,0.00,0.00}{\textbf{#1}}}
\newcommand{\ExtensionTok}[1]{#1}
\newcommand{\FloatTok}[1]{\textcolor[rgb]{0.00,0.00,0.81}{#1}}
\newcommand{\FunctionTok}[1]{\textcolor[rgb]{0.00,0.00,0.00}{#1}}
\newcommand{\ImportTok}[1]{#1}
\newcommand{\InformationTok}[1]{\textcolor[rgb]{0.56,0.35,0.01}{\textbf{\textit{#1}}}}
\newcommand{\KeywordTok}[1]{\textcolor[rgb]{0.13,0.29,0.53}{\textbf{#1}}}
\newcommand{\NormalTok}[1]{#1}
\newcommand{\OperatorTok}[1]{\textcolor[rgb]{0.81,0.36,0.00}{\textbf{#1}}}
\newcommand{\OtherTok}[1]{\textcolor[rgb]{0.56,0.35,0.01}{#1}}
\newcommand{\PreprocessorTok}[1]{\textcolor[rgb]{0.56,0.35,0.01}{\textit{#1}}}
\newcommand{\RegionMarkerTok}[1]{#1}
\newcommand{\SpecialCharTok}[1]{\textcolor[rgb]{0.00,0.00,0.00}{#1}}
\newcommand{\SpecialStringTok}[1]{\textcolor[rgb]{0.31,0.60,0.02}{#1}}
\newcommand{\StringTok}[1]{\textcolor[rgb]{0.31,0.60,0.02}{#1}}
\newcommand{\VariableTok}[1]{\textcolor[rgb]{0.00,0.00,0.00}{#1}}
\newcommand{\VerbatimStringTok}[1]{\textcolor[rgb]{0.31,0.60,0.02}{#1}}
\newcommand{\WarningTok}[1]{\textcolor[rgb]{0.56,0.35,0.01}{\textbf{\textit{#1}}}}
\usepackage{graphicx}
\makeatletter
\def\maxwidth{\ifdim\Gin@nat@width>\linewidth\linewidth\else\Gin@nat@width\fi}
\def\maxheight{\ifdim\Gin@nat@height>\textheight\textheight\else\Gin@nat@height\fi}
\makeatother
% Scale images if necessary, so that they will not overflow the page
% margins by default, and it is still possible to overwrite the defaults
% using explicit options in \includegraphics[width, height, ...]{}
\setkeys{Gin}{width=\maxwidth,height=\maxheight,keepaspectratio}
% Set default figure placement to htbp
\makeatletter
\def\fps@figure{htbp}
\makeatother
\setlength{\emergencystretch}{3em} % prevent overfull lines
\providecommand{\tightlist}{%
  \setlength{\itemsep}{0pt}\setlength{\parskip}{0pt}}
\setcounter{secnumdepth}{-\maxdimen} % remove section numbering
\ifluatex
  \usepackage{selnolig}  % disable illegal ligatures
\fi

\title{apropl.r}
\author{denis}
\date{2021-07-16}

\begin{document}
\maketitle

\begin{Shaded}
\begin{Highlighting}[]
\CommentTok{\#!/usr/bin/r}

\CommentTok{\# For different types of functions, different basis sets may be appropriate.}
\FunctionTok{set.seed}\NormalTok{(}\DecValTok{123}\NormalTok{)}
\NormalTok{A1ps }\OtherTok{\textless{}{-}} \FunctionTok{c}\NormalTok{(}\AttributeTok{mp =} \FloatTok{12.5}\NormalTok{, }\AttributeTok{mc =} \FloatTok{12.3}\NormalTok{, }\AttributeTok{md =} \FloatTok{12.4}\NormalTok{)}
\NormalTok{A1ps  }
\end{Highlighting}
\end{Shaded}

\begin{verbatim}
##   mp   mc   md 
## 12.5 12.3 12.4
\end{verbatim}

\begin{Shaded}
\begin{Highlighting}[]
\NormalTok{A2ps }\OtherTok{\textless{}{-}} \FunctionTok{c}\NormalTok{(}\AttributeTok{pm =} \FloatTok{12.5}\NormalTok{, }\AttributeTok{pm =} \FloatTok{12.3}\NormalTok{, }\AttributeTok{pm =} \FloatTok{12.4}\NormalTok{)}
\NormalTok{A2ps  }
\end{Highlighting}
\end{Shaded}

\begin{verbatim}
##   pm   pm   pm 
## 12.5 12.3 12.4
\end{verbatim}

\begin{Shaded}
\begin{Highlighting}[]
\CommentTok{\# ordered different value compared fireworks update laptop}
\FunctionTok{diff}\NormalTok{(A1ps }\SpecialCharTok{\textgreater{}=}\NormalTok{ A2ps)}
\end{Highlighting}
\end{Shaded}

\begin{verbatim}
## mc md 
##  0  0
\end{verbatim}

\begin{Shaded}
\begin{Highlighting}[]
\CommentTok{\# Polynomials are often used, and there are some standard sets of orthogonal }
\CommentTok{\# polynomials, such as Jacobi, Hermite, and so on.}
\FunctionTok{poly}\NormalTok{(A1ps, A2ps, }\AttributeTok{degree =} \DecValTok{1}\NormalTok{, }\AttributeTok{coefs =} \ConstantTok{NULL}\NormalTok{, }\AttributeTok{raw =} \ConstantTok{FALSE}\NormalTok{, }\AttributeTok{simple =} \ConstantTok{FALSE}\NormalTok{)}
\end{Highlighting}
\end{Shaded}

\begin{verbatim}
##              1.0           0.1
## mp  7.071068e-01  7.071068e-01
## mc -7.071068e-01 -7.071068e-01
## md  9.813078e-17  9.813078e-17
## attr(,"degree")
## [1] 1 1
## attr(,"coefs")
## attr(,"coefs")[[1]]
## attr(,"coefs")[[1]]$alpha
## [1] 12.4
## 
## attr(,"coefs")[[1]]$norm2
## [1] 1.00 3.00 0.02
## 
## 
## attr(,"coefs")[[2]]
## attr(,"coefs")[[2]]$alpha
## [1] 12.4
## 
## attr(,"coefs")[[2]]$norm2
## [1] 1.00 3.00 0.02
## 
## 
## attr(,"class")
## [1] "poly"   "matrix"
\end{verbatim}

\begin{Shaded}
\begin{Highlighting}[]
\CommentTok{\# For periodic functions especially, orthogonal trigonometric }
\CommentTok{\# functions are useful.}
\FunctionTok{trigamma}\NormalTok{(A1ps)}
\end{Highlighting}
\end{Shaded}

\begin{verbatim}
##         mp         mc         md 
## 0.08328522 0.08469517 0.08398428
\end{verbatim}

\begin{Shaded}
\begin{Highlighting}[]
\FunctionTok{trigamma}\NormalTok{(A2ps)}
\end{Highlighting}
\end{Shaded}

\begin{verbatim}
##         pm         pm         pm 
## 0.08328522 0.08469517 0.08398428
\end{verbatim}

\begin{Shaded}
\begin{Highlighting}[]
\CommentTok{\# 2.2.6 Approximation of Vectors In high{-}dimensional vector spaces, it is often }
\CommentTok{\# useful to approximate a given vector in terms of vectors from a lower }
\CommentTok{\# dimensional space.}
\FunctionTok{cov}\NormalTok{(A1ps, A2ps)}
\end{Highlighting}
\end{Shaded}

\begin{verbatim}
## [1] 0.01
\end{verbatim}

\begin{Shaded}
\begin{Highlighting}[]
\CommentTok{\# Suppose, for example, that V ⊂ IR n is a vector space of dimension k }
\CommentTok{\# (necessarily, k ≤ n) and x is a given n{-}vector.}
\NormalTok{k }\OtherTok{=} \DecValTok{5}
\NormalTok{n }\OtherTok{=} \DecValTok{1}

\ControlFlowTok{if}\NormalTok{ (k }\SpecialCharTok{\textless{}}\NormalTok{ n) \{}
    \FunctionTok{print}\NormalTok{(k }\SpecialCharTok{\textless{}}\NormalTok{ n)}
\NormalTok{\} }\ControlFlowTok{else}\NormalTok{ \{}
  \FunctionTok{vector}\NormalTok{(}\AttributeTok{mode =} \StringTok{"logical"}\NormalTok{, }\AttributeTok{length =}\NormalTok{ 0L)}
\NormalTok{\}}
\end{Highlighting}
\end{Shaded}

\begin{verbatim}
## logical(0)
\end{verbatim}

\begin{Shaded}
\begin{Highlighting}[]
\CommentTok{\# pipelines type procedure pass checkup method}
\CommentTok{\# buffer lines ...}
\CommentTok{\# We wish to determine a vector x̃ in V that approximates x.}
\NormalTok{Vx }\OtherTok{\textless{}{-}} \FunctionTok{c}\NormalTok{(}\FunctionTok{c}\NormalTok{(}\FunctionTok{runif}\NormalTok{(n, }\AttributeTok{min =} \DecValTok{0}\NormalTok{, }\AttributeTok{max =} \DecValTok{1}\NormalTok{), }\AttributeTok{open =} \DecValTok{1}\NormalTok{, }
             \AttributeTok{encoding =} \FunctionTok{getOption}\NormalTok{(}\StringTok{"A2ps"}\NormalTok{)))}
\NormalTok{Vx}
\end{Highlighting}
\end{Shaded}

\begin{verbatim}
##                open 
## 0.2875775 1.0000000
\end{verbatim}

\begin{Shaded}
\begin{Highlighting}[]
\CommentTok{\# Optimally of the Fourier Coefficients}
\NormalTok{googledrive}\SpecialCharTok{::}\FunctionTok{as\_id}\NormalTok{(}\StringTok{"http://a2ps.vx"}\NormalTok{)}
\end{Highlighting}
\end{Shaded}

\begin{verbatim}
## [1] NA
## attr(,"class")
## [1] "drive_id"
\end{verbatim}

\begin{Shaded}
\begin{Highlighting}[]
\FunctionTok{attr}\NormalTok{(A2ps, }\AttributeTok{which =} \StringTok{"http://a2ps.vx"}\NormalTok{, }\AttributeTok{exact =} \StringTok{"http://a2ps.vx"}\NormalTok{)}
\end{Highlighting}
\end{Shaded}

\begin{verbatim}
## NULL
\end{verbatim}

\begin{Shaded}
\begin{Highlighting}[]
\FunctionTok{drop}\NormalTok{(n }\SpecialCharTok{+} \DecValTok{1}\NormalTok{)}
\end{Highlighting}
\end{Shaded}

\begin{verbatim}
## [1] 2
\end{verbatim}

\begin{Shaded}
\begin{Highlighting}[]
\FunctionTok{window}\NormalTok{(k }\SpecialCharTok{*}\NormalTok{ n)}
\end{Highlighting}
\end{Shaded}

\begin{verbatim}
## [1] 5
## attr(,"tsp")
## [1] 1 1 1
\end{verbatim}

\begin{Shaded}
\begin{Highlighting}[]
\FunctionTok{attr}\NormalTok{(A2ps, }\AttributeTok{which =} \StringTok{"http://a2ps.vx"}\NormalTok{, }\AttributeTok{exact =} \StringTok{"http://a2ps.vx/tse"}\NormalTok{)}
\end{Highlighting}
\end{Shaded}

\begin{verbatim}
## NULL
\end{verbatim}

\begin{Shaded}
\begin{Highlighting}[]
\CommentTok{\# One obvious criterion would be based on a norm of the difference of the given }
\CommentTok{\# vector and the approximating vector.}
\FunctionTok{outer}\NormalTok{(A2ps, A1ps, }\AttributeTok{FUN =} \StringTok{"*"}\NormalTok{)}
\end{Highlighting}
\end{Shaded}

\begin{verbatim}
##        mp     mc     md
## pm 156.25 153.75 155.00
## pm 153.75 151.29 152.52
## pm 155.00 152.52 153.76
\end{verbatim}

\begin{Shaded}
\begin{Highlighting}[]
\CommentTok{\# lollipop val breadcrumb water}
\NormalTok{loll }\OtherTok{\textless{}{-}} \FunctionTok{c}\NormalTok{(}\AttributeTok{pop =} \FloatTok{2.4}\NormalTok{, }\AttributeTok{bread =} \FloatTok{2.4}\NormalTok{, }\AttributeTok{water =} \FloatTok{2.4}\NormalTok{)}
\NormalTok{loll}
\end{Highlighting}
\end{Shaded}

\begin{verbatim}
##   pop bread water 
##   2.4   2.4   2.4
\end{verbatim}

\begin{Shaded}
\begin{Highlighting}[]
\CommentTok{\# This difference is a truncation error. Let u 1 , . . . , u k be an }
\CommentTok{\# orthonormal basis set for V, and let}
\NormalTok{u1 }\OtherTok{\textless{}{-}} \FunctionTok{trunc}\NormalTok{(loll, A1ps)}
\NormalTok{u1}
\end{Highlighting}
\end{Shaded}

\begin{verbatim}
##   pop bread water 
##     2     2     2
\end{verbatim}

\begin{Shaded}
\begin{Highlighting}[]
\CommentTok{\# where the c i are the Fourier coefficients of x, x, u i . Now l}
\CommentTok{\# et v = a 1 u 1 + · · · + a k u k be any other vector in V, and consider}
\NormalTok{xx }\OtherTok{\textless{}{-}} \FunctionTok{c}\NormalTok{(}\AttributeTok{from =}\NormalTok{ u1, }\AttributeTok{to =}\NormalTok{ loll, }\AttributeTok{strict =} \ConstantTok{TRUE}\NormalTok{)}
\NormalTok{xx}
\end{Highlighting}
\end{Shaded}

\begin{verbatim}
##   from.pop from.bread from.water     to.pop   to.bread   to.water     strict 
##        2.0        2.0        2.0        2.4        2.4        2.4        1.0
\end{verbatim}

\begin{Shaded}
\begin{Highlighting}[]
\CommentTok{\# Therefore we have x − x ≤ x − v, and so x̃ is the best approximation of}
\CommentTok{\# x with respect to the Euclidean norm in the k{-}dimensional vector space V.}
\ControlFlowTok{if}\NormalTok{ (xx }\SpecialCharTok{!=}\NormalTok{ xx }\SpecialCharTok{||}\NormalTok{ xx }\SpecialCharTok{!=}\NormalTok{ xx)\{}
    \FunctionTok{c}\NormalTok{(xx) }\SpecialCharTok{+}\NormalTok{ spacetime}\SpecialCharTok{::}\FunctionTok{geometry}\NormalTok{(}\AttributeTok{obj =}\NormalTok{ u1)}
\NormalTok{\}}
\NormalTok{xx}
\end{Highlighting}
\end{Shaded}

\begin{verbatim}
##   from.pop from.bread from.water     to.pop   to.bread   to.water     strict 
##        2.0        2.0        2.0        2.4        2.4        2.4        1.0
\end{verbatim}

\begin{Shaded}
\begin{Highlighting}[]
\CommentTok{\# Choice of the Best Basis Subset}
\CommentTok{\# Now, posing the problem another way, we may seek the best k{-}dimensional}
\CommentTok{\# subspace of IR n from which to choose an approximating vector.}
\NormalTok{IR }\OtherTok{\textless{}{-}} \FunctionTok{subset.default}\NormalTok{(xx, }\FunctionTok{missing}\NormalTok{(u1), A2ps)}
\NormalTok{IR}
\end{Highlighting}
\end{Shaded}

\begin{verbatim}
## named numeric(0)
\end{verbatim}

\begin{Shaded}
\begin{Highlighting}[]
\CommentTok{\# This question}
\CommentTok{\# is not well{-}posed (because the one{-}dimensional vector space determined by x}
\CommentTok{\# is the solution), but we can pose a related interesting question: suppose we}
\CommentTok{\# have a Fourier expansion of x in terms of a set of n orthogonal basis vectors,}
\CommentTok{\# u 1 , . . . , u n , and we want to choose the “best” k basis vectors from this }
\CommentTok{\# set and use them to form an approximation of x.}
\NormalTok{k }\OtherTok{=}\NormalTok{ n}
\NormalTok{un }\OtherTok{\textless{}{-}} \FunctionTok{c}\NormalTok{(k, }\AttributeTok{color =} \DecValTok{99}\NormalTok{, }\AttributeTok{conf.level =} \FloatTok{0.95}\NormalTok{, }
             \AttributeTok{std =} \DecValTok{12}\NormalTok{, }\AttributeTok{margin =} \FunctionTok{c}\NormalTok{(}\DecValTok{1}\NormalTok{, }\DecValTok{2}\NormalTok{),}
             \AttributeTok{space =} \FloatTok{0.2}\NormalTok{, }\AttributeTok{main =} \ConstantTok{NULL}\NormalTok{, }\AttributeTok{mfrow =} \ConstantTok{NULL}\NormalTok{, }\AttributeTok{mfcol =} \ConstantTok{NULL}\NormalTok{)}
\NormalTok{un}
\end{Highlighting}
\end{Shaded}

\begin{verbatim}
##                 color conf.level        std    margin1    margin2      space 
##       1.00      99.00       0.95      12.00       1.00       2.00       0.20
\end{verbatim}

\begin{Shaded}
\begin{Highlighting}[]
\CommentTok{\# (This restriction of the problem is}
\CommentTok{\# equivalent to choosing a coordinate system.)}
\NormalTok{later}\SpecialCharTok{::}\FunctionTok{create\_loop}\NormalTok{(}\AttributeTok{autorun =} \ConstantTok{FALSE}\NormalTok{)}
\end{Highlighting}
\end{Shaded}

\begin{verbatim}
## <event loop>
##   id: 1
\end{verbatim}

\begin{Shaded}
\begin{Highlighting}[]
\CommentTok{\# We see the solution immediately}
\CommentTok{\# from inequality (2.42): we choose the k u i s corresponding to the k }
\CommentTok{\# largest c i s}
\CommentTok{\# in absolute value, and we take}
\FunctionTok{call}\NormalTok{(}\StringTok{"k"}\NormalTok{, u1, }\FunctionTok{c}\NormalTok{(}\AttributeTok{i =} \FloatTok{2.42}\NormalTok{))}
\end{Highlighting}
\end{Shaded}

\begin{verbatim}
## k(c(pop = 2, bread = 2, water = 2), c(i = 2.42))
\end{verbatim}

\begin{Shaded}
\begin{Highlighting}[]
\FunctionTok{call}\NormalTok{(}\StringTok{"n"}\NormalTok{, xx)}
\end{Highlighting}
\end{Shaded}

\begin{verbatim}
## n(c(from.pop = 2, from.bread = 2, from.water = 2, to.pop = 2.4, 
## to.bread = 2.4, to.water = 2.4, strict = 1))
\end{verbatim}

\begin{Shaded}
\begin{Highlighting}[]
\FunctionTok{outer}\NormalTok{(n, xx, }\AttributeTok{FUN =} \StringTok{"*"}\NormalTok{)}
\end{Highlighting}
\end{Shaded}

\begin{verbatim}
##      from.pop from.bread from.water to.pop to.bread to.water strict
## [1,]        2          2          2    2.4      2.4      2.4      1
\end{verbatim}

\begin{Shaded}
\begin{Highlighting}[]
\CommentTok{\# 2.2.7 Flats, Acne Spaces, and Hyperplanes}
\CommentTok{\# Given an n{-}dimensional vector space of order n, IR n for example, consider a}
\CommentTok{\# system of m linear equations in the n{-}vector variable x,}
\NormalTok{Acne }\OtherTok{\textless{}{-}} \FunctionTok{c}\NormalTok{(n, k, A2ps)}
\NormalTok{Acne}
\end{Highlighting}
\end{Shaded}

\begin{verbatim}
##             pm   pm   pm 
##  1.0  1.0 12.5 12.3 12.4
\end{verbatim}

\begin{Shaded}
\begin{Highlighting}[]
\NormalTok{m }\OtherTok{=} \DecValTok{12}
\NormalTok{n }\OtherTok{=} \DecValTok{1}

\CommentTok{\# where c 1 , . . . , c m are linearly independent n{-}vectors (and hence m ≤ n).}
\NormalTok{c1 }\OtherTok{\textless{}{-}}\NormalTok{ m }\SpecialCharTok{\textless{}}\NormalTok{ n}
\NormalTok{c1}
\end{Highlighting}
\end{Shaded}

\begin{verbatim}
## [1] FALSE
\end{verbatim}

\begin{Shaded}
\begin{Highlighting}[]
\CommentTok{\# The set of points defined by these linear equations is called a rat.}
\NormalTok{PlantGrowth}
\end{Highlighting}
\end{Shaded}

\begin{verbatim}
##    weight group
## 1    4.17  ctrl
## 2    5.58  ctrl
## 3    5.18  ctrl
## 4    6.11  ctrl
## 5    4.50  ctrl
## 6    4.61  ctrl
## 7    5.17  ctrl
## 8    4.53  ctrl
## 9    5.33  ctrl
## 10   5.14  ctrl
## 11   4.81  trt1
## 12   4.17  trt1
## 13   4.41  trt1
## 14   3.59  trt1
## 15   5.87  trt1
## 16   3.83  trt1
## 17   6.03  trt1
## 18   4.89  trt1
## 19   4.32  trt1
## 20   4.69  trt1
## 21   6.31  trt2
## 22   5.12  trt2
## 23   5.54  trt2
## 24   5.50  trt2
## 25   5.37  trt2
## 26   5.29  trt2
## 27   4.92  trt2
## 28   6.15  trt2
## 29   5.80  trt2
## 30   5.26  trt2
\end{verbatim}

\end{document}
